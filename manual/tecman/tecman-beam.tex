\textbf{TODO}
\section{Fully coupled piezoelectric beam}
A fully coupled piezoelectric beam, differently from a piezoelectric beam,
do contribute to the electri equations of the abstrc nodes it is
linked to.
Each abstract node do represent an electrode. The constitutive law
at any given evaluation point is
\begin{equation}
	\cubr{
		\cvvect{
			\T{F}\\
			\T{Q}
		}
	}=\sqbr{
		\matr{cc}{
			\T{E}_{kk} & \T{E}_{kV}\\
			\T{E}_{Vk} & \T{E}_{VV}
		}
	}
	\cubr{
		\cvvect{
			\T{k}\\
			\T{V}
		}
	}
\end{equation}
where $\T{F}$ are the six beam internal actions at the evaluation point, 
$\T{Q}$ the vector of electrodes charges per unit of beam length,
$\T{k}$ the six beam generalized deformations and $\T{V}$ the electric difference of potential
at the electrodes.
The equation contribution to the structural nodes are the same of the piezoelectric beam.
The difference with respect to the piezoelectric beam is the equations contribution to the
asbtract nodes. The unknowns of the abstract nodes do represent the electric difference of potential
at the electrodes. The equations at the abstract nodes states that the sum of charge flux (the electric
current) flowing out from each node must be equal to zero:
\begin{equation}
	\sum_i I_i = 0
\end{equation}
where $I_i$ is the current flowing out from the abstract nodedue to the connected element $i$.
For this element the current flowing out from the electrodes is
\begin{equation}
	\T{I} = -\int_l \dot{\T{Q}} \mathrm{d}s = 
		-\int_l \T{E}_{Vk}\dot{\T{k}}  + \T{E}_{VV} \dot{\T{V}} \mathrm{d}s
		\label{eg:piezobeam-current}
\end{equation}
The fully copled beam element thus turns out to be viscoelastic, as its contribution to the electric
equations do depend on the time derivative of the beam generalized measure.
The integral $\int () \mathrm{d}s$ is computed with the same gauss quadrature scheme
used for the evaluation of the beam stiffness matrix.
The assembled residual contribution is equal to Eq.~\eqref{eg:piezobeam-current} with sign changed.
