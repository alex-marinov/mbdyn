% MBDyn (C) is a multibody analysis code.
% http://www.mbdyn.org
%
% Copyright (C) 1996-2006
%
% Pierangelo Masarati  <masarati@aero.polimi.it>
%
% Dipartimento di Ingegneria Aerospaziale - Politecnico di Milano
% via La Masa, 34 - 20156 Milano, Italy
% http://www.aero.polimi.it
%
% Changing this copyright notice is forbidden.
%
% This program is free software; you can redistribute it and/or modify
% it under the terms of the GNU General Public License as published by
% the Free Software Foundation (version 2 of the License).
% 
%
% This program is distributed in the hope that it will be useful,
% but WITHOUT ANY WARRANTY; without even the implied warranty of
% MERCHANTABILITY or FITNESS FOR A PARTICULAR PURPOSE.  See the
% GNU General Public License for more details.
%
% You should have received a copy of the GNU General Public License
% along with this program; if not, write to the Free Software
% Foundation, Inc., 59 Temple Place, Suite 330, Boston, MA  02111-1307  USA

\chapter{Modal Element FEM File Format}
\label{sec:APP:EL:STRUCT:JOINT:MODAL:FORMAT}

This section describes the format of the FEM input to the \kw{modal}
joint of MBDyn, as resulting from the \kw{femgen} utility.
The usage of that utility has been already detailed 
in Section~\ref{sec:EL:STRUCT:JOINT:MODAL}; in short, it processes
binary output from NASTRAN, as defined by means of appropriate
ALTER files provided with MBDyn sources, into a file that is suitable
for direct input in MBDyn.
Since it is essentially a plain ASCII file, it is straightforward
to generate it from analogous results obtained with a different 
FEM software, from experiments or manually generated from analytical
or numerical models of any kind.

The format is:
{\small
\begin{verbatim}
<comments>
** RECORD GROUP 1,<any comment; "HEADER">
<comment>
<REV> <NNODES> <NNORMAL> <NATTACHED> <NCONSTRAINT> <NREJECTED>
<comments; NMODES = NNORMAL + NATTACHED + NCONSTRAINT - NREJECTED>
** RECORD GROUP 2,<any comment; "FINITE ELEMENT NODE LIST">
<FEMLABEL> [...<NNODES>]
<comments; FEM label list: <NNODES> integers>
** RECORD GROUP 3,<any comment; "INITIAL MODAL DISPLACEMENTS">
<MODEDISP> [..<NMODES>]
<comments; initial mode displacements: <NMODES> reals>
** RECORD GROUP 4,<any comment; "INITIAL MODAL VELOCITIES">
<MODEVEL> [..<NMODES>]
<comments; initial mode velocities: <NMODES> reals>
** RECORD GROUP 5,<any comment; "NODAL X COORDINATES">
<FEMX> [...<NNODES>]
<comments; FEM node X coordinates>
** RECORD GROUP 6,<any comment; "NODAL Y COORDINATES">
<FEMY> [...<NNODES>]
<comments; FEM node Y coordinates>
** RECORD GROUP 7,<any comment; "NODAL Z COORDINATES">
<FEMZ> [...<NNODES>]
<comments; FEM node Z coordinates>
** RECORD GROUP 8,<any comment; "NON-ORTHOGONALIZED MODE SHAPES">
<comment; NORMAL MODE SHAPE # 1>
    <FEM1 X> <FEM1 Y> <FEM1 Z> <FEM1 RX> <FEM1 RY> <FEM1 RZ>
    <... NNODES>
<comment; NORMAL MODE SHAPE # 2>
    <FEM1 X> <FEM1 Y> <FEM1 Z> <FEM1 RX> <FEM1 RY> <FEM1 RZ>
    <... NNODES>
<... NMODES>
<comments; for each MODE, for each NODE: modal displacements/rotations>
** RECORD GROUP 9,<any comment; "MODAL MASS MATRIX. COLUMN-MAJOR FORM">
<M_1_1>      <...> <M_1_NMODES>
<...>
<M_NMODES_1> <...> <M_NMODES_NMODES>
<comments; the modal mass matrix in column-major (symmetric?)>
** RECORD GROUP 10,<any comment; "MODAL STIFFNESS MATRIX. COLUMN-MAJOR FORM">
<K_1_1>      <...> <K_1_NMODES>
<...>
<K_NMODES_1> <...> <K_NMODES_NMODES>
<comments; the modal stiffness matrix in column-major (symmetric?)>
** RECORD GROUP 11,<any comment; "DIAGONAL OF LUMPED MASS MATRIX">
<M_1_X> <M_1_Y> <M_1_Z> <M_1_RX> <M_1_RY> <M_1_RZ>
<...>
<M_NNODES_X> <...> <M_NNODES_RZ>
<comments; the lumped diagonal mass matrix of the FEM model>
\end{verbatim}
}

An arbitrary number of comment lines may appear within blocks;
the beginning of a block is marked
\begin{verbatim}
** RECORD GROUP <DD>
\end{verbatim}
where the number \kw{<DD>} indicates what block is being read.
The size of each block, i.e.\ the number of values that are expected,
is defined based on the header block, so MBDyn should be able to detect
incomplete or mis-formatted files.

The blocks contain:
\begin{itemize}
\item \kw{RECORD GROUP 1}, a.k.a.\ the ``header'', contains a summary
of the contents of the file:
	\begin{itemize}
	\item \kw{<REV>} is a string that indicates the revision number;
	it is currently ignored;
	\item \kw{<NNODES>} is the number of (exposed) FEM nodes 
	in the FEM model;
	\item \kw{<NNORMAL>} is the number of normal modes;
	\item \kw{<NATTACHED>} is the number of ``attached'',
	i.e.\ static, modes;
	\item \kw{<NCONSTRAINT>} is the number of constraint modes;
	\item \kw{<NREJECTED>} is the number of rejected modes.
	\end{itemize}
Currently, the number of available modes is computed as
\begin{displaymath}
	\kw{NMODES} = \kw{NNORMAL} + \kw{NATTACHED}
		+ \kw{NCONSTRAINT} - \kw{NREJECTED}
\end{displaymath}
because modes are treated in a generalized manner,
so there is no need to consider the different types of shapes
in a specific manner.
Typically, one should put all numbers to 0, except for
\kw{<NNORMAL>} which should be set equal to the number of modes
actually present in the data set.
Remember that MBDyn can still select a subset of the available 
modes to be used in the analysis, so that there is no need 
to further edit this file.

\item \kw{RECORD GROUP 2} contains a listing of the \kw{<NNODES>} labels
of the (exposed) FEM nodes.

\item \kw{RECORD GROUP 3} contains the initial values of the \kw{<NMODES>}
modal unknowns;

\item \kw{RECORD GROUP 4} contains the initial values of the \kw{<NMODES>}
modal unknowns derivatives;

\item \kw{RECORD GROUP 5} contains the $X$ component of the position
of the \kw{<NNODES>} FEM nodes in the reference frame attached 
to the \kw{modal} node (or to the \kw{origin node}, if given).

\item \kw{RECORD GROUP 6} contains the $Y$ component of the data above;

\item \kw{RECORD GROUP 7} contains the $Z$ component of the data above;

\item \kw{RECORD GROUP 8} contains the non-orthogonalized components 
of the \kw{<NMODES>} modes; for each mode, the three components 
$X$, $Y$, $Z$ of the modal displacement, and the three components
$RX$, $RY$, $RZ$ of the linearized modal rotations are listed;
each mode shape is separated by a comment line, which typically is
\begin{verbatim}
**    NORMAL MODE SHAPE #  <N>
\end{verbatim}
for readability;

\item \kw{RECORD GROUP 9} contains the modal mass matrix,
i.e.\ a square, \kw{NMODES} times \kw{NMODES} matrix that contains
the reduced mass $\sqbr{m}$ resulting from the multiplication
\begin{equation}
	\label{eq:modal-mass-constraint}
	\sqbr{m} = \cubr{X}^T \sqbr{M} \cubr{X}
\end{equation}
When only normal modes are used, it is diagonal.
It can be semi-definite positive, or even zero, if a partially 
or entirely static model is considered.

\item \kw{RECORD GROUP 10} contains the modal stiffness matrix,
i.e.\ a square, \kw{NMODES} times \kw{NMODES} matrix that contains
the reduced stiffness $\sqbr{k}$ resulting from the multiplication
\begin{displaymath}
	\sqbr{k} = \cubr{X}^T \sqbr{K} \cubr{X}
\end{displaymath}
When only normal modes are used, it is diagonal; in that case,
the diagonal contains the modal mass times the square 
of the eigenvalues, i.e.\ $k_{ii} = \omega_i^2 m_{ii}$.
It should be definite positive; in fact, rigid degrees of freedom
that would make it semi-definite should rather be modeled by combining
separate modal submodels by way of multibody connections, so that
the multibody capability to handle finite relative displacements
and rotations is exploited.

\item \kw{RECORD GROUP 11} contains the lumped inertia matrix
associated to the \kw{<NNODES>} (exposed) FEM nodes;
for each node, the $X$, $Y$, $Z$, $RX$, $RY$ and $RY$ inertia
coefficients are listed.
The resulting diagonal matrix must satisfy the constraint
illustrated in Equation~\ref{eq:modal-mass-constraint}.
\end{itemize}

Although the format loosely requires that no more than 6 numerical values 
appear on a single line, MBDyn is very forgiving about this and can parse
the input regardless of the formatting within each block.

As an example, a very simple, hand-made FEM model file is presented below.
It models a FEM model made of three aligned nodes, where inertia 
is only associated to the mid-node.
As a consequence, the three mode shapes must be interpreted as static
shapes, since the modal mass matrix is null.
Note that each line is prefixed with a two-digit line number 
that is not part of the iput file.
Also, for readability, all comments are prefixed by ``\kw{**}''
like the mandatory ``\kw{** RECORD GROUP}'' lines, although not required 
by the format of the file.

{\small
\begin{verbatim}
01  ** MBDyn MODAL DATA FILE
02  ** NODE SET "ALL" 
03    
04    
05  ** RECORD GROUP 1, HEADER
06  **   REVISION,  NODE,  NORMAL, ATTACHMENT, CONSTRAINT, REJECTED MODES.
07        REV0         3         3         0         0         0
08  **
09  ** RECORD GROUP 2, FINITE ELEMENT NODE LIST
10       1001 1002 1003
11  
12  **
13  ** RECORD GROUP 3, INITIAL MODAL DISPLACEMENTS
14   0 0 0
15  **
16  ** RECORD GROUP 4, INITIAL MODALVELOCITIES
17   0 0 0
18  **
19  ** RECORD GROUP 5, NODAL X COORDINATES
20   0
21   0
22   0
23  **
24  ** RECORD GROUP 6, NODAL Y COORDINATES
25  -2.
26   0
27   2.
28  **
29  ** RECORD GROUP 7, NODAL Z COORDINATES
30   0
31   0
32   0
33  **
34  ** RECORD GROUP 8, MODE SHAPES
35  **    NORMAL MODE SHAPE #  1
36  0 0 1 0 0 0
37  0 0 0 0 0 0
38  0 0 1 0 0 0
39  **    NORMAL MODE SHAPE #  2
40  1 0 0 0 0 0
41  0 0 0 0 0 0
42  1 0 0 0 0 0
43  **    NORMAL MODE SHAPE #  3
44  0 1 0 0 0 0
45  0 0 0 0 0 0
46  0 -1 0 0 0 0
47  **
48  ** RECORD GROUP 9, MODAL MASS MATRIX
49  0 0 0
50  0 0 0
51  0 0 0
52  **
53  ** RECORD GROUP 10, MODAL STIFFNESS MATRIX
54  1 0   0
55  0 1e2 0
56  0 0   1e6
57  **
58  ** RECORD GROUP 11, DIAGONAL OF LUMPED MASS MATRIX
59  0 0 0 0 0 0
60  1 1 1 1 1 1
61  0 0 0 0 0 0
\end{verbatim}
}
