% MBDyn (C) is a multibody analysis code.
% http://www.mbdyn.org
%
% Copyright (C) 1996-2009
%
% Pierangelo Masarati  <masarati@aero.polimi.it>
%
% Dipartimento di Ingegneria Aerospaziale - Politecnico di Milano
% via La Masa, 34 - 20156 Milano, Italy
% http://www.aero.polimi.it
%
% Changing this copyright notice is forbidden.
%
% This program is free software; you can redistribute it and/or modify
% it under the terms of the GNU General Public License as published by
% the Free Software Foundation (version 2 of the License).
% 
%
% This program is distributed in the hope that it will be useful,
% but WITHOUT ANY WARRANTY; without even the implied warranty of
% MERCHANTABILITY or FITNESS FOR A PARTICULAR PURPOSE.  See the
% GNU General Public License for more details.
%
% You should have received a copy of the GNU General Public License
% along with this program; if not, write to the Free Software
% Foundation, Inc., 59 Temple Place, Suite 330, Boston, MA  02111-1307  USA
%
% Alessandro Fumagalli <fumagalli@aero.polimi.it> is the author of this document

Aerodynamic elements apply aerodynamic forces to structural nodes.

This section is not intended 
to give details about the aerodynamic models adopted but mainly 
discuss the computation of the contributions to the Jacobian 
matrix of the aerodynamic elements.

\paragraph{Files.} \
It is implemented in files\\
\begin{tabular}{l}
\texttt{mbdyn/aero/aeroelem.h} \\
\texttt{mbdyn/aero/aeroelem.cc} 
\end{tabular}

\section{Linearization of Sectional Aerodynamic Forces and Moments}
MBDyn's built-in 2D aerodynamics computes
aerodynamic forces $\tilde{\T{f}}_{a/\xi}$
and moments $\tilde{\T{c}}_{a/\xi}$ per unit span,
in a relative frame at station $\xi$,
based on the instantaneous value of linear and angular velocity
boundary conditions, respectively $\tilde{\T{v}}$ and $\tilde{\T{\omega}}$,
expressed in the same relative frame, namely
\begin{subequations}
\begin{align}
	\tilde{\T{f}}_{a/\xi}
	&=
	\tilde{\T{f}}_{a/\xi}\plbr{\tilde{\T{v}}, \tilde{\T{\omega}}, \xi}
	\\
	\tilde{\T{c}}_{a/\xi}
	&=
	\tilde{\T{c}}_{a/\xi}\plbr{\tilde{\T{v}}, \tilde{\T{\omega}}, \xi}
	.
\end{align}
\end{subequations}

The boundary conditions are computed by projecting the \emph{effective}
linear and angular velocity at station $\xi$,
respectively $\T{v}\plbr{\xi}$ and $\T{\omega}\plbr{\xi}$,
in the reference frame of the aerodynamic forces,
namely
\begin{subequations}
\begin{align}
	\tilde{\T{v}}\plbr{\xi}
	&=
	\TT{R}^T\plbr{\xi} \T{v}\plbr{\xi}
	\\
	\tilde{\T{\omega}}\plbr{\xi}
	&=
	\TT{R}^T\plbr{\xi} \T{\omega}\plbr{\xi}
	,
\end{align}
\end{subequations}
where $\TT{R}\plbr{\xi}$ is the matrix that expresses
the local orientation of the aerodynamic section at station $\xi$.

In detail, the effective linear velocity at an arbitrary station
is the combination of the absolute velocity resulting from the kinematics
of the model, of an airstream velocity that may depend
on the absolute location of a reference point and on time,
and of a contribution resulting from an inflow model, namely
\begin{align}
	\T{v}\plbr{\xi}
	&=
	\T{v}_{\text{kin}}\plbr{\xi}
	+ \T{v}_{\infty}\plbr{\T{x}\plbr{\xi}, t}
	+ \T{v}_{\text{inflow}}\plbr{\T{x}\plbr{\xi}}
	.
\end{align}
It is assumed that the last two contributions do not depend
on the state of the problem, or only depend on it in a loose manner,
and thus do not directly participate in the linearization
of the aerodynamic forces and moments.

Their linearization yields
\begin{subequations}
\begin{align}
	\delta\tilde{\T{v}}
	&=
	\TT{R}^T\plbr{\xi} \plbr{
		\delta\T{v}_{\text{kin}}\plbr{\xi}
		+ \T{v}\plbr{\xi} \times \T{\theta}_{\delta}\plbr{\xi}
	}
	\\
	\delta\tilde{\T{\omega}}
	&=
	\TT{R}^T\plbr{\xi} \plbr{
		\delta\T{\omega}\plbr{\xi}
		+ \T{\omega}\plbr{\xi} \times \T{\theta}_{\delta}\plbr{\xi}
	}
	,
\end{align}
\end{subequations}
where
\begin{align}
	\T{\theta}_{\delta}\plbr{\xi}
	&=
	\text{ax}\plbr{\delta\TT{R}\plbr{\xi} \TT{R}^T\plbr{\xi}}
	.
\end{align}

By means of numerical integration, the force and moment per unit span
are integrated into discrete contributions to the force and the moment
applied to the appropriate node equilibrium.
This is usually done by multiplying each force and moment per unit span
contribution by an appropriate reference length coefficient.
As a consequence, in the following force and moment contributions
will be considered, namely
\begin{subequations}
\begin{align}
	\tilde{\T{f}}_a
	&=
	\tilde{\T{f}}_a\plbr{\tilde{\T{v}}, \tilde{\T{\omega}}, \xi}
	\\
	\tilde{\T{c}}_a
	&=
	\tilde{\T{c}}_a\plbr{\tilde{\T{v}}, \tilde{\T{\omega}}, \xi}
	.
\end{align}
\end{subequations}
As already mentioned, each weighted sectional force and moment contribution
is applied to the appropriate node after projection in the global reference
frame.
The contribution to the force and moment equilibrium of the $n$-th node is
\begin{subequations}
	\label{eq:aero:fm-contrib}
\begin{align}
	\Delta\T{f}_n
	&=
	\TT{R}\plbr{\xi} \tilde{\T{f}}_a
	\label{eq:aero:fm-contrib:f}
	\\
	\Delta\T{c}_n
	&=
	\TT{R}\plbr{\xi} \tilde{\T{c}}_a
	+ \plbr{\T{x}\plbr{\xi} - \T{x}_n} \times \Delta\T{f}_n
	\label{eq:aero:fm-contrib:m}
	.
\end{align}
\end{subequations}
Their linearization yields
\begin{subequations}
\begin{align}
	\delta\Delta\T{f}_n
	&=
	- \Delta\T{f}_n \times \T{\theta}_{\delta}\plbr{\xi}
	+ \TT{R}\plbr{\xi} \delta\tilde{\T{f}}_a
	\\
	\delta\Delta\T{c}_n
	&=
	- \Delta\T{c}_n \times \T{\theta}_{\delta}\plbr{\xi}
	+ \TT{R}\plbr{\xi} \delta\tilde{\T{c}}_a
	- \Delta\T{f}_n \times \plbr{
		\delta\T{x}\plbr{\xi}
		- \delta\T{x}_n
	}
	\nonumber \\
	&\hphantom{= } \mbox{}
	+ \plbr{
		\T{x}\plbr{\xi}
		- \T{x}_n
	} \times \plbr{
		- \Delta\T{f}_n \times \T{\theta}_{\delta}\plbr{\xi}
		+ \TT{R}\plbr{\xi} \delta\tilde{\T{f}}_a
	}
	,
\end{align}
\end{subequations}
which can be summarized as
\begin{align}
	\cubr{\cvvect{
		\delta\Delta\T{f}_n \\
		\delta\Delta\T{c}_n
	}}
	&=
	\sqbr{\cvvect{
		- \Delta\T{f}_n \times{} \\
		- \Delta\T{c}_n \times{}
		- \plbr{\T{x}\plbr{\xi} - \T{x}_n} \times \Delta\T{f}_n \times{}
	}} \T{\theta}_{\delta}\plbr{\xi}
	\nonumber \\
	& \hphantom{= } \mbox{}
	+ \sqbr{\cvvect{
		\TT{0} \\
		- \Delta\T{f}_n \times{}
	}} \plbr{
		\delta\T{x}\plbr{\xi}
		- \delta\T{x}_n
	}
	\nonumber \\
	& \hphantom{= } \mbox{}
	+ \sqbr{\matr{cc}{
		\TT{I} & \TT{0} \\
		\plbr{\T{x}\plbr{\xi} - \T{x}_n} \times{} & \TT{I}
	}} \cubr{\cvvect{
		\TT{R}\plbr{\xi} \delta\tilde{\T{f}}_a \\
		\TT{R}\plbr{\xi} \delta\tilde{\T{c}}_a
	}}
	.
\end{align}

It is assumed that the Jacobian matrix of the sectional force and moment
with respect to the linear and angular velocity
is either available or can be computed by numerical differentiation.
The resulting force and moment perturbation is
\begin{align}
	\cubr{\cvvect{
		\delta\tilde{\T{f}}_a \\
		\delta\tilde{\T{c}}_a
	}} &= \sqbr{\matr{cc}{
		\tilde{\T{f}}_{a/\tilde{\T{v}}} & \tilde{\T{f}}_{a/\tilde{\T{\omega}}} \\
		\tilde{\T{c}}_{a/\tilde{\T{v}}} & \tilde{\T{c}}_{a/\tilde{\T{\omega}}}
	}} \cubr{\cvvect{
		\delta\tilde{\T{v}} \\
		\delta\tilde{\T{\omega}}
	}}
	\label{eq:aero:jac}
	.
\end{align}
The computation of the matrix of Eq.~(\ref{eq:aero:jac})
is delegated to the \texttt{AeroData} class.

Each type of element determines how the sectional force and moment
contributions are applied to the nodes, and how the sectional
boundary conditions at each section are computed from the kinematics
of the nodes.



\section{Numerical Linearization of Sectional Forces}
Consider an arbitrary submatrix of the Jacobian matrix
of Eq.~(\ref{eq:aero:jac}), $\TT{J}=\T{p}_{/\T{q}}$.
Its generic element, the $c$-th component of $\T{p}$
derived by the $r$-th component of $\T{q}$, is
\begin{align}
	J_{rc} &= \frac{\partial\T{p}_r}{\partial\T{q}_c}
	.
\end{align}
A forward difference approach is used, namely
\begin{align}
	J_{rc} &\cong \frac{
		\T{p}_r\plbr{\T{q} + \Delta q \T{e}_c}
		- \T{p}_r\plbr{\T{q}}
	}{
		\Delta q
	}
	,
\end{align}
where $\T{e}_c$ is the unit vector along the $c$-th component,
and $\Delta q$ is a suitably chosen perturbation.
Alternatively, a centered difference approach can be used, namely
\begin{align}
	J_{rc} &\cong \frac{
		\T{p}_r\plbr{\T{q} + \Delta q \T{e}_c}
		- \T{p}_r\plbr{\T{q} - \Delta q \T{e}_c}
	}{
		2 \Delta q
	}
	.
\end{align}
The perturbation is
\begin{align}
	\Delta q &= \varepsilon \nrbr{\T{q}} + \nu
	.
\end{align}
Since the boundary condition $\T{q}$ is perturbed in order to determine
an equivalent perturbation of angle of attack, a resolution of few tenth
of degree is deemed sufficient.
As a consequence, $\varepsilon>0$ must be a ``small'' number
that, in case $\T{e}_c$ is orthogonal to $\T{q}$,
yields an angle of the order of a tenth of a degree.
The default value is $\varepsilon=10^{-3}$.
However, in order to avoid divisions by too small numbers,
the perturbation is corrected by another ``small'' parameter,
$\nu>0$.
The default value is $\nu=10^{-9}$.



\section{Aerodynamic body}
The \texttt{aerodynamic body} element applies aerodynamic forces
to the structural node it is connected to,
based on the relative velocity between an aerodynamic surface attached 
to the node and the airstream.

The boundary conditions are related to the rigid body motion
of the node, so
\begin{subequations}
\begin{align}
	\TT{R}\plbr{\xi}
	&=
	\TT{R}_n \TT{R}_a \TT{R}_t\plbr{\xi}
	\\
	\T{b}\plbr{\xi}
	&=
	\TT{R}_n\plbr{
		\tilde{\T{b}}_0
		+ \TT{R}_a \plbr{
			b\plbr{\xi} \T{e}_1
			+ \xi \T{e}_3
		}
	}
	\\
	\T{\omega}\plbr{\xi}
	&=
	\T{\omega}_n
	\\
	\T{v}_{\text{kin}}\plbr{\xi}
	&=
	\T{v}_n + \T{\omega}_n \times \T{b}\plbr{\xi}
	,
\end{align}
\end{subequations}
where
$\TT{R}_n$ is the orientation of the node,
$\TT{R}_a$ is the relative orientation of the aerodynamics
with respect to the node,
$\TT{R}_t$ is the pretwist matrix,
$\T{v}_n$ is the absolute velocity of the node,
$\T{\omega}_n$ is the absolute angular velocity of the node,
$\tilde{\T{b}}_0$ is an offset between the node and the reference location
of the aerodynamic body,
and $b\plbr{\xi}$ is the chordwise location of the point
where the boundary conditions are evaluated.

Their linearization is straightforward:
\begin{subequations}
\begin{align}
	\T{\theta}_{\delta}\plbr{\xi}
	&=
	\T{\theta}_{n\delta}
	\\
	\delta\T{b}\plbr{\xi}
	&=
	- \T{b}\plbr{\xi}\times\T{\theta}_{n\delta}
	\\
	\delta\T{\omega}\plbr{\xi}
	&=
	\delta\T{\omega}_n
	\\
	\delta\T{v}_{\text{kin}}\plbr{\xi}
	&=
	\delta\T{v}_n
	- \T{b}\plbr{\xi}\times\delta\T{\omega}_n
	- \T{\omega}_n \times \T{b}\plbr{\xi} \times \T{\theta}_{n\delta}
	.
\end{align}
\end{subequations}
Eq.~(\ref{eq:aero:fm-contrib:m}) can be rewritten as
\begin{align}
	\Delta\T{c}_n
	&=
	\TT{R}\plbr{\xi}\plbr{
		\tilde{\T{c}}_a
		+ \tilde{\T{o}}\plbr{\xi} \times \tilde{\T{f}}_a
	} ,
\end{align}
where
$\tilde{\T{o}}\plbr{\xi}=\TT{R}^T\plbr{\xi}\plbr{\T{x}\plbr{\xi} - \T{x}_n}$,
the offset between the point where the force is applied and the node, 
in the reference frame of the node, does not depend on the kinematics
of the system, since the body is rigid.
Its linearization yields
\begin{align}
	\delta\Delta\T{c}_n
	&=
	- \Delta\T{c}_n \times \T{\theta}_{\delta}\plbr{\xi}
	+ \TT{R}\plbr{\xi} \plbr{
		\delta\tilde{\T{c}}_a
		+ \tilde{\T{o}}\plbr{\xi} \times \delta\tilde{\T{f}}_a
	}
	,
\end{align}
since $\delta\tilde{\T{o}}\plbr{\xi}\equiv{0}$.
So the linearized force and moment is
\begin{align}
	\cubr{\cvvect{
		\delta\Delta\T{f}_n \\
		\delta\Delta\T{c}_n
	}}
	&=
	- \sqbr{\cvvect{
		\Delta\T{f}_n \times{} \\
		\Delta\T{c}_n \times{}
	}} \T{\theta}_{n\delta}
	+ \sqbr{\matr{cc}{
		\TT{I} & \TT{0} \\
		\T{o}\plbr{\xi}\times{} & \TT{I}
	}} \cubr{\cvvect{
		\TT{R}\plbr{\xi} \delta\tilde{\T{f}}_a \\
		\TT{R}\plbr{\xi} \delta\tilde{\T{c}}_a
	}}
	.
\end{align}
with $\T{o}\plbr{\xi}=\TT{R}\plbr{\xi} \tilde{\T{o}}\plbr{\xi}$.

The linearization of the boundary conditions yields
\begin{subequations}
\begin{align}
	\cubr{\cvvect{
		\delta\tilde{\T{v}}\plbr{\xi} \\
		\delta\tilde{\T{\omega}}\plbr{\xi}
	}}
	&=
	\sqbr{\matr{cc}{
		\TT{R}^T\plbr{\xi} & \TT{0} \\
		\TT{0} & \TT{R}^T\plbr{\xi}
	}} \lplbr{
		\sqbr{\cvvect{
			\T{v}\plbr{\xi}\times{}
				- \T{\omega}_n\times\T{b}\plbr{\xi}\times{}
			\\
			\T{\omega}_n\times{}
		}} \T{\theta}_{n\delta}
	} \nonumber \\
	& \hphantom{= } \rplbr{
		\mbox{} + \sqbr{\matr{cc}{
			\TT{I} & -\T{b}\plbr{\xi}\times{} \\
			\TT{0} & \TT{I}
		}} \cubr{\cvvect{
			\delta\T{v}_n \\
			\delta\T{\omega}_n
		}}
	}
	\\
	&\equu 
	\sqbr{\matr{cc}{
		\TT{R}^T\plbr{\xi} & \TT{0} \\
		\TT{0} & \TT{R}^T\plbr{\xi}
	}} \lplbr{
		\sqbr{\cvvect{
			\plbr{\T{v}\plbr{\xi} + \T{b}\plbr{\xi}\times \T{\omega}_n}\times{}
			\\
			\T{0}
		}} \delta\T{g}_n
	} \nonumber \\
	& \hphantom{= } \rplbr{
		\mbox{} + \sqbr{\matr{cc}{
			\TT{I} & -\T{b}\plbr{\xi}\times{} \\
			\TT{0} & \TT{I}
		}} \cubr{\cvvect{
			\delta\dot{\T{x}}_n \\
			\delta\dot{\T{g}}_n
		}}
	}
	.
\end{align}
\end{subequations}
After defining
\begin{subequations}
\begin{align}
	\T{f}_{a/\tilde{\T{v}}} &= \TT{R}\plbr{\xi} \tilde{\T{f}}_{a/\tilde{\T{v}}} \TT{R}^T\plbr{\xi} \\
	\T{f}_{a/\tilde{\T{\omega}}} &= \TT{R}\plbr{\xi} \tilde{\T{f}}_{a/\tilde{\T{\omega}}} \TT{R}^T\plbr{\xi} \\
	\T{c}_{a/\tilde{\T{v}}} &= \TT{R}\plbr{\xi} \tilde{\T{c}}_{a/\tilde{\T{v}}} \TT{R}^T\plbr{\xi} \\
	\T{c}_{a/\tilde{\T{\omega}}} &= \TT{R}\plbr{\xi} \tilde{\T{c}}_{a/\tilde{\T{\omega}}} \TT{R}^T\plbr{\xi}
	,
\end{align}
\end{subequations}
the linearization becomes
\begin{align}
	\sqbr{\matr{c}{
		\T{f}_{a/\tilde{\T{v}}}
		\\
		\T{o}\plbr{\xi}\times\T{f}_{a/\tilde{\T{v}}} + \T{c}_{a/\tilde{\T{v}}}
	}} \delta\dot{\T{x}}_n
	\nonumber \\
	+ \sqbr{\matr{c}{
		\T{f}_{a/\tilde{\T{\omega}}} - \T{f}_{a/\tilde{\T{v}}} \T{b}\plbr{\xi}\times{}
		\\
		\T{o}\plbr{\xi} \times \T{f}_{a/\tilde{\T{\omega}}}
		+ \T{c}_{a/\tilde{\T{\omega}}}
		- \plbr{
			\T{o}\plbr{\xi} \times \T{f}_{a/\tilde{\T{v}}}
			+ \T{c}_{a/\tilde{\T{v}}}
		}
		\T{b}\plbr{\xi}\times{}
	}} \delta\dot{\T{g}}_n
	\nonumber \\
	+ \sqbr{\matr{c}{
		\T{f}_{a/\tilde{\T{v}}} \plbr{\T{v}\plbr{\xi} + \T{b}\plbr{\xi} \times \T{\omega}_n} \times{} \\
		\plbr{
			\T{o}\plbr{\xi} \times \T{f}_{a/\tilde{\T{v}}}
			+ \T{c}_{a/\tilde{\T{v}}}
		} \plbr{\T{v}\plbr{\xi} + \T{b}\plbr{\xi} \times \T{\omega}_n} \times{}
	}} \delta\T{g}_n
	\nonumber \\
	- \sqbr{\cvvect{
		\Delta\T{f}_n \times{} \\
		\Delta\T{c}_n \times{} \\
	}} \delta\T{g}_n
	\equu
	\cubr{\cvvect{
		\delta\Delta\T{f}_n \\
		\delta\Delta\T{c}_n
	}}
	.
\end{align}



\begin{comment}
\section{Aerodynamic body}
The \texttt{aerodynamic body} element applies aerodynamic forces
to the structural node it is connected to,
based on the relative velocity between an aerodynamic surface attached 
to the node and the airstream.

The aerodynamic forces and moments acting on the attached node are:
\begin{subequations}
\begin{align}
	\T{f} &= \TT{R}_{\text{loc}} \T{f}_a\\ 
	\T{c} &= \TT{R}_{\text{loc}} \T{c}_a + \T{x}_r\times\T{f} 
\end{align}
\end{subequations}
where $\TT{R}_{\text{loc}} = \TT{R}_n\TT{R}_a\TT{R}_t$ is the orientation 
matrix from the local frame of the aerodynamics to the global, 
while $\T{f}_a$ and $\T{c}_a$ are the aerodynamic force and moment 
respectively in the reference frame of the aerodynamics. $\TT{R}_n$
is the orientation matrix of the node, $\TT{R}_a$ is the
relative orientation matrix of the aerodynamics frame with respect 
to the node frame while $\TT{R}_t$ is a rotation matrix 
related to the twist of the aerodynamic surface.
Both $\TT{R}_a$ and $\TT{R}_t$ do not depend on the unknowns
of the problem, and thus their perturbation is zero.
The vector $\T{x}_r = \TT{R}_n\T{b}$ is the offset of the aerodynamic
surface with respect to the node; $\T{b}$ is constant.

The aerodynamic forces and moments are computed as functions 
of a velocity vector in the reference frame of the aerodynamics:
\begin{equation}\label{eq:fa=g(va)}
	\sqbr{\matr{c}{\T{f_a}\\\T{c_a}}} = \T{g}\plbr{\T{V}_{a}}
\end{equation}
where $\T{V}_{a} = \sqbr{\T{v}_a^T\ \T{\omega}_a^T}^T$ includes both the 
linear and angular velocity in the reference frame of the aerodynamics:
\begin{equation}
	\T{V}_{a} = \sqbr{\matr{c}{
		\TT{R}^T_{\text{loc}} \plbr{\T{v}_n + \T{\omega}_n\times\T{x}_r - \T{v}_{\infty} - \T{v}_{\text{ind}}}\\
		\TT{R}^T_{\text{loc}} \T{\omega}_n
		}}
\end{equation}
The total airstream velocity in the aerodynamic frame of reference 
includes also the free stream velocity $\T{v}_{\infty}$ and
the induced velocity $\T{v}_{\text{ind}}$, were the aerodynamic element
connected to any induced velocity model.
These two contributions will not be considered in the following
since they do not depend on the unknowns of the problem\footnote{In general,
induced velocity models depend on the aerodynamic loads,
and thus may indirectly depend on the unknowns of the problem.
This dependence is neglected.
}, and thus do not contribute to the Jacobian matrix of the problem. 

The perturbation of the aerodynamic force in the global reference frame
yields:
\begin{align}
	\delta\T{f}
	& = \T{\theta}_{\delta}\times\TT{R}_{\text{loc}}\T{f}_a
	+ \TT{R}_{\text{loc}}\frac{\partial\T{f}_a}{\partial\T{V}_a}
		\delta\T{V}_a
	\nonumber \\
	&= \T{\theta}_{\delta}\times \T{f}
	+ \TT{R}_{\text{loc}}\frac{\partial\T{f}_a}{\partial\T{V}_a}
		\delta\T{V}_a
\end{align}
The $3\times6$ matrix $\partial\T{f}_a/\partial\T{V}_a$, 
named $\T{J_f}_a$ in the
following, is computed numerically using the finite difference method
described below.

The perturbation of the aerodynamic moment in the global reference frame
yields:
\begin{align}
	\delta\T{c}
	&= \T{\theta}_{\delta}\times\TT{R}_{\text{loc}}\T{c}_a
	+ \TT{R}_{\text{loc}}\frac{\partial\T{c}_a}{\partial\T{V}_a}
		\delta\T{V}_a
	+ \delta\T{x}_r\times\T{f}
	+ \T{x}_r\times\delta\T{f}
	\nonumber \\
	&= \T{\theta}_{\delta}\times\T{c}
	+ \TT{R}_{\text{loc}}\frac{\partial\T{c}_a}{\partial\T{V}_a}
		\delta\T{V}_a
	+ \T{x}_r\times
		\TT{R}_{\text{loc}}\frac{\partial\T{f}_a}{\partial\T{V}_a}
		\delta\T{V}_a
	\nonumber \\
	&= \T{\theta}_{\delta}\times\T{c}
	+ \TT{R}_{\text{loc}}\plbr{
		\frac{\partial\T{c}_a}{\partial\T{V}_a}
		+ \T{b}\times\frac{\partial\T{f}_a}{\partial\T{V}_a}
	} \delta\T{V}_a
\end{align}
The $3\times6$ matrix $\partial\T{c}_a/\partial\T{V}_a$, 
named $\T{J_c}_a$ in the
following, is computed numerically as well, using a finite difference method.

The Jacobian matrix of the aerodynamic forces and moments in the reference 
frame of the aerodynamics is computed computed numerically,
using a forward finite difference method involving a numerical perturbation
of the relative velocity in the reference frame of the aerodynamics.
In detail, each component of the $6\times6$ matrix $\T{J}_a$ is computed as:
\begin{equation}
	{J_a}_{ij} = \frac{g_i(\T{V}_a + \Delta\T{V}_a) 
		- g_i(\T{V}_a)}{\Delta {V_a}_j}
\end{equation}
The perturbation of the velocity, $\Delta\T{V}_a$ in this case, plays
a crucial role. The procedure here adopted consists in obtaining this
perturbation based on the norm of the linear velocity $\T{v}_a$ for what 
concerns the perturbation of the linear velocity components, based on
the norm of the angular velocity $\T{\omega}_a$ for what concerns 
the angular velocity components. Thus:
\begin{equation}
	\Delta {V_a}_j = \Bigl\{\matr{c}{
		\varepsilon\lvert\T{v}_a\lvert\text{ if $j = 1,2,3$}\\
		\varepsilon\lvert\T{\omega}_a\lvert\text{ if $j = 4,5,6$}
		}
\end{equation}
To summarize the procedure, it can be helpful to report the algorithm
actually implemented:
\begin{verbatim}
F0a = GetForces(V0a);
for(i = 1; i <= 6; i++)	{
    deltaVa = V0a; 
				
    if (i <= 3)	{
        delta = norm(V0a) * epsilon;
    } else		{
        delta = norm(W0a) * epsilon;
    }
    deltaVa(i) = V0a(i) + delta;
					
    Fa = GetForces(deltaVa);
	
    for(j = 1; j <= 6; j++)	{
        Ja(j,i) = (Fa(j) - Fa0(j)) / delta 
    }
}
\end{verbatim}

For ease of notation, in the following the $6\times6$ matrix $\T{J}_a$ is 
divided into four sub-matrices
\begin{equation}\label{eq:JaSub}
	\T{J}_a = \sqbr{\matr{c}{\T{J_f}_a\\\T{J_c}_a}}
	=\sqbr{\matr{cc}{\T{J}_{a11} & \T{J}_{a12}\\ \T{J}_{a21} & \T{J}_{a22}}}
\end{equation}

To correctly express the Jacobian matrix contributions, the perturbation 
$\delta\T{V}_a$ and $\delta \T{x}_r$ must be expressed in terms of 
the node coordinates and velocities, namely:
\begin{align}
	\delta\T{V}_a &= \delta\sqbr{\matr{c}{
		\TT{R}^T_{\text{loc}}
			% \plbr{\T{v}_n + \T{\omega}_n\times\T{x}_r - \T{v}_{\infty} - \T{v}_{\text{ind}}}
			\T{v}_{\text{rel}}
		\\
		\TT{R}^T_{\text{loc}} \T{\omega}_n
		}}
	\nonumber \\
	&= \sqbr{\matr{c}{
	\TT{R}^T_{\text{loc}} \plbr{
		\T{v}_{\text{rel}} \times \T{\theta}_\delta
		+ \delta\T{v}_n
		+ \delta\T{\omega}_n\times\T{x}_r
		- \T{\omega}_n\times\T{x}_r\times\T{\theta}_{\delta}}
	\\
	\TT{R}^T_{\text{loc}}\plbr{
		\T{\omega}_n \times\T{\theta}_\delta
		+ \delta\T{\omega}_n
	}
	}}
\end{align}
with $\T{v}_{\text{rel}}=\T{v}_n + \T{\omega}_n\times\T{x}_r - \T{v}_{\infty} - \T{v}_{\text{ind}}$
and exploiting $\delta\T{x}_r = \T{\theta}_\delta\times\T{x}_r$.

After some manipulation, the perturbation of the aerodynamic forces  
in the global reference frame reads:
\begin{align}
	\delta\T{f} &= 
	-\TT{R}_{\text{loc}}\T{f}_a\times\T{\theta}_\delta
	\nonumber \\
	&\hphantom{= }+ \TT{R}_{\text{loc}}\T{J}_{a11}\TT{R}^T_{\text{loc}}\T{v}_{\text{rel}}\times\T{\theta}_\delta
	\nonumber \\
	&\hphantom{= }+ \TT{R}_{\text{loc}}\T{J}_{a11}\TT{R}^T_{\text{loc}}\delta\T{v}_n
	\nonumber \\
	&\hphantom{= }- \TT{R}_{\text{loc}}\T{J}_{a11}\TT{R}^T_{\text{loc}}
		\T{x}_r\times\delta\T{\omega}_n
	\nonumber \\
	&\hphantom{= }- \TT{R}_{\text{loc}}\T{J}_{a11}\TT{R}^T_{\text{loc}}
		\T{\omega}_n\times\T{x}_r\times\T{\theta}_\delta
	\nonumber \\
	&\hphantom{= }+ \TT{R}_{\text{loc}}\T{J}_{a12}\TT{R}^T_{\text{loc}}\T{\omega}_n
		\times\T{\theta}_\delta
	\nonumber \\
	&\hphantom{= }+ \TT{R}_{\text{loc}}\T{J}_{a12}\TT{R}^T_{\text{loc}}\delta\T{\omega}_n
	\nonumber \\
	&=
	\plbr{
		- \T{f}\times{}
		+ \TT{R}_{\text{loc}}\T{J}_{a11}\TT{R}^T_{\text{loc}}\plbr{
			\T{v}_{\text{rel}}\times{}
			- \T{\omega}_n\times\T{x}_r\times{}
		}
		+ \TT{R}_{\text{loc}}\T{J}_{a12}\TT{R}^T_{\text{loc}}\T{\omega}_n\times{}
	} \T{\theta}_\delta
	\nonumber \\
	&\hphantom{= }+ \TT{R}_{\text{loc}}\T{J}_{a11}\TT{R}^T_{\text{loc}}\delta\T{v}_n
	\nonumber \\
	&\hphantom{= }+ \plbr{
		- \TT{R}_{\text{loc}}\T{J}_{a11}\TT{R}^T_{\text{loc}} \T{x}_r\times{}
		+ \TT{R}_{\text{loc}}\T{J}_{a12}\TT{R}^T_{\text{loc}}
	} \delta\T{\omega}_n
	\label{eq:deltaF}
\end{align}
while the perturbation of the aerodynamic moments yields:

\begin{align}
	\delta\T{c} &= 
	- \TT{R}_{\text{loc}}\T{c}_a\times\T{\theta}_\delta
	\nonumber \\
	&\hphantom{= }+ \TT{R}_{\text{loc}}\T{J}_{a21}\TT{R}^T_{\text{loc}}\T{v}_{\text{rel}}\times\T{\theta}_\delta
	\nonumber \\
	&\hphantom{= }+ \TT{R}_{\text{loc}}\T{J}_{a21}\TT{R}^T_{\text{loc}}\delta\T{v}_n
	\nonumber \\
	&\hphantom{= }- \TT{R}_{\text{loc}}\T{J}_{a21}\TT{R}^T_{\text{loc}}
		\T{x}_r\times\delta\T{\omega}_n
	\nonumber \\
	&\hphantom{= }- \TT{R}_{\text{loc}}\T{J}_{a21}\TT{R}^T_{\text{loc}}
		\T{\omega}_n\times\T{x}_r\times\T{\theta}_\delta
	\nonumber \\
	&\hphantom{= }+ \TT{R}_{\text{loc}}\T{J}_{a22}\TT{R}^T_{\text{loc}}\T{\omega}_n
		\times\T{\theta}_\delta
	\nonumber \\
	&\hphantom{= }+ \TT{R}_{\text{loc}}\T{J}_{a22}\TT{R}^T_{\text{loc}}\delta\T{\omega}_n
	\nonumber \\
	&\hphantom{= }+ \T{f}\times\T{x}_r\times\T{\theta}_\delta
	\nonumber \\
	&\hphantom{= }+ \T{x}_r \times \delta\T{f}
	\nonumber \\
	&=
	\lplbr{
		- \T{c}\times{}
		+ \TT{R}_{\text{loc}}\T{J}_{a21}\TT{R}^T_{\text{loc}} \plbr{
			\T{v}_{\text{rel}}\times{}
			- \T{\omega}_n\times\T{x}_r\times{}
		}
		+ \TT{R}_{\text{loc}}\T{J}_{a22}\TT{R}^T_{\text{loc}}\T{\omega}_n\times{}
	}
	\nonumber \\
	&\hphantom{= }\rplbr{\mbox{\hspace{3mm}} + \T{x}_r \times 
	\plbr{
		\TT{R}_{\text{loc}}\T{J}_{a11}\TT{R}^T_{\text{loc}}\plbr{
			\T{v}_{\text{rel}}\times{}
			- \T{\omega}_n\times\T{x}_r\times{}
		}
		+ \TT{R}_{\text{loc}}\T{J}_{a12}\TT{R}^T_{\text{loc}}\T{\omega}_n\times{}
	}} \T{\theta}_\delta
	\nonumber \\
	&\hphantom{= }+ \plbr{
		\TT{R}_{\text{loc}}\T{J}_{a21}\TT{R}^T_{\text{loc}}
		+ \T{x}_r \times \TT{R}_{\text{loc}}\T{J}_{a11}\TT{R}^T_{\text{loc}}
	} \delta\T{v}_n
	\nonumber \\
	&\hphantom{= }+ \lplbr{
		- \TT{R}_{\text{loc}}\T{J}_{a21}\TT{R}^T_{\text{loc}}\T{x}_r\times{}
		+ \TT{R}_{\text{loc}}\T{J}_{a22}\TT{R}^T_{\text{loc}}
	}
	\nonumber \\
	&\hphantom{= } \rplbr{
		\mbox{\hspace{5mm}} + \T{x}_r \times \plbr{
		- \TT{R}_{\text{loc}}\T{J}_{a11}\TT{R}^T_{\text{loc}} \T{x}_r\times{}
		+ \TT{R}_{\text{loc}}\T{J}_{a12}\TT{R}^T_{\text{loc}}
	}} \delta\T{\omega}_n
\end{align}
where the last term, which involves $\delta\T{f}$, is the same of
Eq.~(\ref{eq:deltaF}).
\end{comment}




\section{Aerodynamic Beam (3 Nodes)}
The \texttt{aerodynamic beam3} element applies aerodynamic forces
to the nodes of a three node structural beam element. 

The aerodynamic forces/moments acting on each node 
are computed based on the relative velocity of a set of locations
along the beam and the airstream.
The kinematic quantities of the beam are computed based
on an interpolation of the kinematics of the three nodes.

\paragraph{Kinematics Interpolation.}
\emph{Note: this part is common to all elements that use the
three-node beam discretization and interpolation model.}
The generic field variable $\T{p}\plbr{x}$ is interpolated
using parabolic functions related to the value of the field 
variable at three locations that in general can be offset
from the nodes,
\begin{align}
	\T{p}\plbr{\xi}
	&=
	\sum_{i=1,2,3} N_i\plbr{\xi} \T{p}_i
	,
\end{align}
with
\begin{subequations}
\begin{align}
	N_1 &= \frac{1}{2} \xi \plbr{\xi - 1}
	\\
	N_2 &= 1 - \xi^2
	\\
	N_3 &= \frac{1}{2} \xi \plbr{\xi + 1}
	.
\end{align}
\end{subequations}

Orientation at an arbitrary location $\xi$:
\begin{align}
	\TT{R}\plbr{\xi}
	&=
	\TT{R}_{2_a} \TT{R}\plbr{\T{g}\plbr{\xi}}
	\\
	\T{g}\plbr{\xi}
	&=
	\sum_{i=1,3} N_i\plbr{\xi} \T{g}_{2\leftarrow i}
	= N_1\plbr{\xi} \T{g}_{2\leftarrow 1} + N_3\plbr{\xi} \T{g}_{2\leftarrow 3}
	\\
	\T{g}_{2\leftarrow i}
	&=
	\text{Cay}\plbr{\TT{R}_{2_a}^T \TT{R}_{i_a}}
	.
\end{align}
Orientation is dealt with specially, given its special nature.
The orientation of the mid node is used as a reference,
and the orientation parameters that express the relative orientation
between each of the end nodes and the mid node are interpolated.
The interpolated orientation parameters are used to compute
the interpolated relative orientation matrix, which is then
pre-multiplied by the orientation matrix of the mid node.
Summation in this case occurs on $i=1,3$ only because by definition
$\T{g}_{2\leftarrow 2}=\text{Cay}(\TT{I})\equiv\T{0}$.

Since the Cayley-Gibbs-Rodrigues orientation parameters are used,
the magnitude of the relative orientation between each end node
and the mid node must be limited (formally, to $\pi/2$,
but it should be less for accuracy).

Position at an arbitrary location $\xi$:
\begin{align}
	\T{x}\plbr{\xi}
	&=
	\sum_{i=1,2,3} N_i\plbr{\xi} \plbr{
		\T{x}_i + \T{o}_i
	}
	,
\end{align}
with $\T{o}_i=\TT{R}_i \tilde{\T{o}}_i$.

Angular velocity at an arbitrary location $\xi$:
\begin{align}
	\T{\omega}\plbr{\xi}
	&=
	\sum_{i=1,2,3} N_i\plbr{\xi} \T{\omega}_i
	.
\end{align}

Velocity at an arbitrary location $\xi$:
\begin{align}
	\T{v}_{\text{kin}}\plbr{\xi}
	&=
	\sum_{i=1,2,3} N_i\plbr{\xi} \plbr{
		\T{v}_i
		+ \T{\omega}_i \times \T{o}_i
	}
	.
\end{align}

\paragraph{Perturbation of Interpolated Kinematics.}
Orientation perturbation at an arbitrary location $\xi$:
\begin{align}
	\delta\T{g}\plbr{\xi}
	&=
	\sum_{i=1,3} N_i\plbr{\xi} \delta\T{g}_{2\leftarrow i}
	\\
	\delta\T{g}_{2\leftarrow i}
	&=
	\TT{\Gamma}^{-1}\plbr{\T{g}_{2\leftarrow i}} \T{\theta}_{(2\leftarrow i) \delta}
	\\
	\T{\theta}_{(2\leftarrow i) \delta}
	&=
	\TT{R}_{2_a}^T \plbr{
		\T{\theta}_{i\delta}
		- \T{\theta}_{2\delta}
	}
	\\
	\T{\theta}_{\delta}\plbr{\xi}
	&=
	\T{\theta}_{2\delta}
	+ \sum_{i=1,3} \TT{R}_{2_a} \TT{\Gamma}\plbr{\T{g}\plbr{\xi}}
		N_i\plbr{\xi} \TT{\Gamma}^{-1}\plbr{\T{g}_{2\leftarrow i}}
		\TT{R}_{2_a}^T \plbr{\T{\theta}_{i\delta} - \T{\theta}_{2\delta}}
	\nonumber \\
	&= \plbr{
		\TT{I}
		- \TT{R}_{2_a} \TT{\Gamma}\plbr{\T{g}\plbr{\xi}}
			\sum_{i=1,3} N_i\plbr{\xi} \TT{\Gamma}^{-1}\plbr{\T{g}_{2\leftarrow i}}
			\TT{R}_{2_a}^T
	} \T{\theta}_{2\delta}
	\nonumber \\
	& \hphantom{= } \mbox{}
	+ \TT{R}_{2_a} \TT{\Gamma}\plbr{\T{g}\plbr{\xi}}
		N_1\TT{\Gamma}^{-1}\plbr{\T{g}_{2\leftarrow 1}}
		\TT{R}_{2_a}^T \T{\theta}_{1\delta}
	\nonumber \\
	& \hphantom{= } \mbox{}
	+ \TT{R}_{2_a} \TT{\Gamma}\plbr{\T{g}\plbr{\xi}}
		N_3\TT{\Gamma}^{-1}\plbr{\T{g}_{2\leftarrow 3}}
		\TT{R}_{2_a}^T \T{\theta}_{3\delta}
	\nonumber \\
	&=
	\sum_{i=1,2,3} \TT{\Theta}_i\plbr{\xi} \T{\theta}_{i\delta}
	,
\end{align}
with
\begin{subequations}
\begin{align}
	\TT{\Theta}_i\plbr{\xi}
	&=
	\TT{R}_{2_a} \TT{\Gamma}\plbr{\T{g}\plbr{\xi}}
		N_i\TT{\Gamma}^{-1}\plbr{\T{g}_{2\leftarrow i}}
		\TT{R}_{2_a}^T
		& i=\text{1 and 3}
	\\
	\TT{\Theta}_2\plbr{\xi}
	&=
	\TT{I} - \TT{\Theta}_1\plbr{\xi} - \TT{\Theta}_3\plbr{\xi}
\end{align}
\end{subequations}
playing the role of shape functions.

In fact, note that, when $N_1=1$ and $N_3=0$, then
$\TT{\Theta}_1=\TT{I}$, $\TT{\Theta}_2=\TT{\Theta}_3=\TT{0}$
and $\T{\theta}_{\delta}\plbr{\xi}=\T{\theta}_{1\delta}$,
while, when $N_1=0$ and $N_3=1$, then
$\TT{\Theta}_1=\TT{\Theta}_2=\TT{0}$, $\TT{\Theta}_3=\TT{I}$
and $\T{\theta}_{\delta}\plbr{\xi}=\T{\theta}_{3\delta}$.
Finally, when $N_1=N_3=0$, then
$\TT{\Theta}_1=\TT{\Theta}_3=\TT{0}$, $\TT{\Theta}_2=\TT{I}$
and $\T{\theta}_{\delta}\plbr{\xi}=\T{\theta}_{2\delta}$.
Moreover, $\sum_{i=1,2,3}\TT{\Theta}_i\plbr{\xi}=\TT{I} \ \forall\xi$.

Position perturbation at an arbitrary location $\xi$:
\begin{align}
	\delta\T{x}\plbr{\xi}
	&=
	\sum_{i=1,2,3} N_i\plbr{\xi} \plbr{
		\delta\T{x}_i + \T{\theta}_{i\delta}\times\T{o}_i
	}
	.
\end{align}

Angular velocity perturbation at an arbitrary location $\xi$:
\begin{align}
	\delta\T{\omega}\plbr{\xi}
	&=
	\sum_{i=1,2,3} N_i \delta\T{\omega}_i
	\nonumber \\
	&\equu
	\sum_{i=1,2,3} N_i \plbr{
		\delta\dot{\T{g}}_i
		- \T{\omega}_i \times \delta\T{g}_i
	}
	.
\end{align}

Velocity perturbation at an arbitrary location $\xi$:
\begin{align}
	\delta\T{v}_{\text{kin}}\plbr{\xi}
	&=
	\sum_{i=1,2,3} N_i\plbr{\xi} \plbr{
		\delta\T{v}_i
		+ \delta\T{\omega}_i \times \T{o}_i
		+ \T{\omega}_i \times \delta\T{o}_i
	}
	\nonumber \\
	&= 
	\sum_{i=1,2,3} N_i\plbr{\xi} \plbr{
		\delta\T{v}_i
		- \T{o}_i \times \delta\T{\omega}_i
		- \T{\omega}_i \times \T{o}_i \times \T{\theta}_{i\delta}
	}
	\nonumber \\
	&\equu
	\sum_{i=1,2,3} N_i\plbr{\xi} \plbr{
		\delta\dot{\T{x}}_i
		- \T{o}_i \times \delta\dot{\T{g}}_i
		- \plbr{\T{\omega}_i \times \T{o}_i} \times \delta\T{g}_i
	}
	.
\end{align}


\paragraph{Boundary Conditions Perturbation.}
Angular velocity perturbation:
\begin{align}
	\delta\tilde{\T{\omega}}\plbr{\xi}
	&= \TT{R}^T\plbr{\xi} \sum_{i=1,2,3}\plbr{
		N_i\plbr{\xi} \delta\T{\omega}_i
		+ \T{\omega}\plbr{\xi} \times \TT{\Theta}_i\plbr{\xi} \T{\theta}_{i\delta}
	}
	\nonumber \\
	&\equu
	\TT{R}^T\plbr{\xi} \plbr{
		N_i\plbr{\xi} \delta\dot{\T{g}}_i
		+ \plbr{
			\T{\omega}\plbr{\xi} \times \TT{\Theta}_i\plbr{\xi}
			- N_i\plbr{\xi} \T{\omega}_i \times{}
		} \delta\T{g}_i
	}
	.
\end{align}

Velocity perturbation:
\begin{align}
	\delta\tilde{\T{v}}\plbr{\xi}
	&=
	\TT{R}^T\plbr{\xi} \lplbr{
		N_i\plbr{\xi} \delta\T{v}_i
		- N_i\plbr{\xi} \T{o}_i \times \delta\T{\omega}_i
	}
	\nonumber \\
	& \hphantom{= \TT{R}^T\plbr{\xi}(}
	\rplbr{
		\mbox{} + \plbr{
			\T{v}\plbr{\xi} \times \TT{\Theta}_i\plbr{\xi}
			- N_i\plbr{\xi} \T{\omega}_i \times \T{o}_i \times{}
		} \T{\theta}_{i\delta}
	}
	\nonumber \\
	&\equu
	\TT{R}^T\plbr{\xi} \lplbr{
		N_i\plbr{\xi} \delta\dot{\T{x}}_i
		- N_i\plbr{\xi} \T{o}_i \times \delta\dot{\T{g}}_i
	}
	\nonumber \\
	& \hphantom{= \TT{R}^T\plbr{\xi}(}
	\rplbr{
		\mbox{} + \plbr{
			\T{v}\plbr{\xi} \times \TT{\Theta}_i\plbr{\xi}
			- N_i\plbr{\xi} \plbr{\T{\omega}_i \times \T{o}_i} \times{}
		} \delta\T{g}_i
	}
	.
\end{align}
They can be summarized as
\begin{align}
	\cubr{\cvvect{
		\delta\tilde{\T{v}}\plbr{\xi} \\
		\delta\tilde{\T{\omega}}\plbr{\xi}
	}}
	&=
	\sqbr{\matr{cc}{
		\TT{R}^T\plbr{\xi} & \TT{0} \\
		\TT{0} & \TT{R}^T\plbr{\xi}
	}} \sum_{i=1,2,3} \lplbr{
		N_i\plbr{\xi} \sqbr{\matr{cc}{
			\TT{I} & - \T{o}_i \times{} \\
			\TT{0} & \TT{I}
		}} \cubr{\cvvect{
			\delta\T{v}_i \\
			\delta\T{\omega}_i
		}}
	}
	\nonumber \\
	& \hphantom{= } \mbox{}
	+ \rplbr{
		\sqbr{\cvvect{
			\T{v}\plbr{\xi} \times \TT{\Theta}_i\plbr{\xi}
			- N_i\plbr{\xi} \T{\omega}_i \times \T{o}_i \times{} \\
			\T{\omega}\plbr{\xi} \times \TT{\Theta}_i\plbr{\xi}
		}} \T{\theta}_{i\delta}
	}
	\nonumber \\
	&\equu
	\sqbr{\matr{cc}{
		\TT{R}^T\plbr{\xi} & \TT{0} \\
		\TT{0} & \TT{R}^T\plbr{\xi}
	}} \sum_{i=1,2,3} \lplbr{
		N_i\plbr{\xi} \sqbr{\matr{cc}{
			\TT{I} & - \T{o}_i \times{} \\
			\TT{0} & \TT{I}
		}} \cubr{\cvvect{
			\delta\dot{\T{x}}_i \\
			\delta\dot{\T{g}}_i
		}}
	}
	\nonumber \\
	& \hphantom{= } \mbox{}
	+ \rplbr{
		\sqbr{\cvvect{
			\T{v}\plbr{\xi} \times \TT{\Theta}_i\plbr{\xi}
			- N_i\plbr{\xi} \plbr{\T{\omega}_i \times \T{o}_i} \times{} \\
			\T{\omega}\plbr{\xi} \times \TT{\Theta}_i\plbr{\xi}
			- N_i\plbr{\xi} \T{\omega}_i \times{}
		}} \delta\T{g}_i
	}
	.
\end{align}

\paragraph{Contribution to Jacobian Matrix.}
After defining
\begin{subequations}
\begin{align}
	\T{f}_{a/\tilde{\T{v}}} &= \TT{R}\plbr{\xi} \tilde{\T{f}}_{a/\tilde{\T{v}}} \TT{R}^T\plbr{\xi} \\
	\T{f}_{a/\tilde{\T{\omega}}} &= \TT{R}\plbr{\xi} \tilde{\T{f}}_{a/\tilde{\T{\omega}}} \TT{R}^T\plbr{\xi} \\
	\T{c}_{a/\tilde{\T{v}}} &= \TT{R}\plbr{\xi} \tilde{\T{c}}_{a/\tilde{\T{v}}} \TT{R}^T\plbr{\xi} \\
	\T{c}_{a/\tilde{\T{\omega}}} &= \TT{R}\plbr{\xi} \tilde{\T{c}}_{a/\tilde{\T{\omega}}} \TT{R}^T\plbr{\xi} \\
	\TT{B}_{\tilde{\T{v}}} &= \T{v}\plbr{\xi} \times \TT{\Theta}_i\plbr{\xi}
		- N_i\plbr{\xi} \plbr{\T{\omega}_i \times \T{o}_i} \times{} \\
	\TT{B}_{\tilde{\T{\omega}}} &= \T{\omega}\plbr{\xi} \times \TT{\Theta}_i\plbr{\xi}
		- N_i\plbr{\xi} \T{\omega}_i \times{} \\
	\T{d}_n\plbr{\xi} &= \T{x}\plbr{\xi} - \T{x}_n
	,
\end{align}
\end{subequations}
the contribution of the $i$-th node's motion to the equilibrium
of the $n$-th node is
\begin{align}
	N_i\plbr{\xi} \sqbr{\matr{c}{
		\T{f}_{a/\tilde{\T{v}}} \\
		\T{d}_n\plbr{\xi} \times \T{f}_{a/\tilde{\T{v}}}
		+ \T{c}_{a/\tilde{\T{v}}}
	}} \delta\dot{\T{x}}_i
	\nonumber \\
%
	\mbox{}
	+ N_i\plbr{\xi} \sqbr{\matr{c}{
		\T{f}_{a/\tilde{\T{\omega}}}
		- \T{f}_{a/\tilde{\T{v}}} \T{o}_i \times{} \\
		\T{d}_n\plbr{\xi} \times \T{f}_{a/\tilde{\T{\omega}}}
		+ \T{c}_{a/\tilde{\T{\omega}}}
		- \plbr{
			\T{d}_n\plbr{\xi} \times \T{f}_{a/\tilde{\T{v}}}
			+ \T{c}_{a/\tilde{\T{v}}}
		} \T{o}_i \times{}
	}} \delta\dot{\T{g}}_i
	\nonumber \\
%
	\mbox{}
	+ \sqbr{\matr{c}{
		\T{f}_{a/\tilde{\T{v}}} \TT{B}_{\tilde{\T{v}}}
			+ \T{f}_{a/\tilde{\T{\omega}}} \TT{B}_{\tilde{\T{\omega}}} \\
		\T{d}_n\plbr{\xi} \times \plbr{
			\T{f}_{a/\tilde{\T{v}}} \TT{B}_{\tilde{\T{v}}}
				+ \T{f}_{a/\tilde{\T{\omega}}} \TT{B}_{\tilde{\T{\omega}}}
		}
		+ \T{c}_{a/\tilde{\T{v}}} \TT{B}_{\tilde{\T{v}}}
			+ \T{c}_{a/\tilde{\T{\omega}}} \TT{B}_{\tilde{\T{\omega}}}
	}} \delta\T{g}_i
	\nonumber \\
%
	\mbox{}
	+ \sqbr{\matr{c}{
		\TT{0} \\
		- \plbr{
			N_i - \delta_{ni}
		} \Delta\T{f}_n \times{}
	}} \delta\T{x}_i
	\nonumber \\
%
	\mbox{}
	+ \sqbr{\matr{c}{
		- \Delta\T{f}_n \times \TT{\Theta}_i\plbr{\xi} \\
		- \plbr{
			\T{d}_n\plbr{\xi} \times \Delta\T{f}_n \times{}
			+ \Delta\T{c}_n \times{}
		} \TT{\Theta}_i\plbr{\xi}
		+ \Delta\T{f}_n \times N_i \T{o}_i \times{}
	}} \delta\T{g}_i
	\nonumber \\
%
	\equu
	\cubr{\cvvect{
		\delta\Delta\T{f}_n \\
		\delta\Delta\T{c}_n
	}}
	,
\end{align}
where $\delta_{ni}$ is Dirac's delta, which is 1 when $i=n$, and 0 otherwise.

\section{Aerodynamic Beam (2 Nodes)}



\begin{comment}
\section{Aerodynamic Beam}

The \texttt{aerodynamic beam} element applies aerodynamic forces
to the nodes of a structural beam element. 
The aerodynamic forces/moments acting on a single node 
are computed based on a relative 
velocity between an aerodynamic surface attached 
to the node and the airstream. By itself, the relative velocity 
used on each node, as well as the position of the aerodynamic 
surface, is computed based on a weighted mean of the velocities
of the (three) beam nodes. 

\newcommand{\Rloc}{{\TT{R}_{\text{loc}}}}
\newcommand{\Rtilde}[1]{\tilde{\TT{R}}_{#1}}
\newcommand{\Cay}[1]{\text{Cay}\plbr{#1}}
\newcommand{\Rot}[1]{\text{Rot}\plbr{#1}}
The aerodynamic forces and moments acting on the $i$-th node of the
beam are:
\begin{subequations}\label{eq:fici}
\begin{align}
	\T{f}_i &= \Rloc {\T{f}_a}_i\\ 
	\T{c}_i &= \Rloc {\T{c}_a}_i 
		+ \plbr{{\T{x}_r}_i - \T{x}_i}\times{\T{f}}_i 
\end{align}
\end{subequations}
where $\Rloc$ is the orientation 
matrix from the local frame of the aerodynamics
to the global, while ${\T{f}_a}_i$ and ${\T{c}_a}_i$ are the 
aerodynamic force and moment 
respectively in the reference frame of the aerodynamics, $\T{x}_r$
is the offset. 
By defining $\T{g}_{2\leftarrow{1}} = \Cay{\Rtilde{2}^T\Rtilde{1}}$ 
and $\T{g}_{2\leftarrow{3}} = \Cay{\Rtilde{2}^T\Rtilde{3}}$ (where, respectively,
$\Rtilde{i} = \TT{R}_i\TT{R}_{ai}$ is the orientation matrix from the 
local frame of the aerodynamics on the $i$-th node to the global),
matrix $\Rloc$ writes:
\begin{equation}
	\Rloc = \Rtilde{2} \Rot{N_1\T{g}_{2\leftarrow{1}} + N_3\T{g}_{2\leftarrow{3}}}
\end{equation}
where $N_1$ and $N_3$ are weight coefficients which depend on
the node the computation is referred to\footnote{Since the 
weight coefficients $N_i$ depends on the node for which the
forces are computed, also $\Rloc$ differs from node to node
even if it is not explicitly indicated in the formulas. }.

The relative velocity of the aerodynamic surface in the global
frame is computed as:
\begin{align}
	\T{v}_r &= \sum_{i=1}^3 N_i \plbr{\T{v}_i 
		+ \T{\omega}_i\times\T{R}_i\T{b}_i}\\
	\T{\omega}_r &= \sum_{i=1}^3 N_i \T{\omega}_i 
\end{align}
which is an interpolation of the velocities of the three nodes
of the beam.  
The total airstream velocity in the aerodynamic frame of reference 
includes also the free stream velocity $\T{v}_{\infty}$ and
the induced velocity $\T{v}_{\text{ind}}$, were the aerodynamic element
connected to any induced velocity model.
These two contributions will not be considered in the following
since they do not depend on the unknowns of the problem.  
In the same fashion, the offset of the reference 
aerodynamic surface is computed as:
\begin{equation}
	\T{x}_r = \sum_{i=1}^3 N_i \plbr{\T{x}_i 
		+ \T{R}_i\T{b}_i}
\end{equation}
The aerodynamic forces and moments $\T{f}_a$ and $\T{c}_a$
are computed using the same algorithm of Eq.~(\ref{eq:fa=g(va)}), 
and its differentiation is computed numerically using the same 
algorithm described in the previous section and its Jacobian matrix
($\T{J}_{a}$) is still divided into four sub-matrices, as in 
Eq.~(\ref{eq:JaSub}).

Differentiating Eq.~(\ref{eq:fici}), one obtains:
\begin{align}
	\delta\T{f}_i &= \delta\Rloc \T{f}_a 
		+ \Rloc \sqbr{\T{J}_{a11}\ \T{J}_{a12}}
		\sqbr{\matr{c}{
		\delta\Rloc^T\T{v}_r + \Rloc^T\delta\T{v}_r\\
		\delta\Rloc^T\T{\omega}_r + \Rloc^T\delta\T{\omega}_r}}\\
\nonumber\delta\T{c}_i &= \delta\Rloc \T{c}_a 
		+ \Rloc \sqbr{\T{J}_{a21}\ \T{J}_{a22}}
		\sqbr{\matr{c}{
		\delta\Rloc^T\T{v}_r + \Rloc^T\delta\T{v}_r\\
		\delta\Rloc^T\T{\omega}_r + \Rloc^T\delta\T{\omega}_r}}\\
	&+ \plbr{\delta\T{x}_r - \delta\T{x}_i}\times\T{f}_i 
		+ \plbr{\T{x}_r - \T{x}_i}\times\delta\T{f}_i
\end{align}
These equations are composed by many terms and, in order to keep the
notation light, each contributions is written separately.
First of all, the most recursive terms are the product between 
$\delta\Rloc$ and a vector, and the product between $\delta\Rloc^T$
and a vector. These, defining $\overline{\T{g}}=N_1\T{g}_{2\leftarrow{1}} 
+ N_3\T{g}_{2\leftarrow{3}}$, 
respectively result in:
\newcommand{\vv}{\T{\overline{v}}}
\begin{align}
\nonumber	\plbr{\delta\Rloc}\vv &= 
		-\plbr{\Rloc\vv}\times\T{\theta_{2\delta}}\\ 
\nonumber	&+\plbr{\Rloc\vv}\times
		\Rtilde{2}\T{\Gamma}\plbr{\overline{\T{g}}}\plbr{
			N_1 \T{\Gamma}^{-1}\plbr{\T{g}_{2\leftarrow{1}}}\Rtilde{2}^T
			\plbr{\T{\theta_{2\delta}}-\T{\theta_{1\delta}}}
			+N_3 \T{\Gamma}^{-1}\plbr{\T{g}_{2\leftarrow{3}}}\Rtilde{2}^T
			\plbr{\T{\theta_{2\delta}}-\T{\theta_{3\delta}}}
		} \\
\nonumber	\plbr{\delta\Rloc^T}\vv &= 
		\Rloc^T\vv\times\T{\theta_{2\delta}}\\ 
\nonumber	&+N_1\sqbr{\Rot{\overline{\T{g}}}}^T
			\plbr{\Rtilde{2}^T\vv}\times
		\T{\Gamma}\plbr{\overline{\T{g}}}
		\T{\Gamma}^{-1}\plbr{\T{g}_{2\leftarrow{1}}}\Rtilde{2}^T
		\plbr{\T{\theta_{1\delta}}-\T{\theta_{2\delta}}}\\ 
\nonumber	&+N_3\sqbr{\Rot{\overline{\T{g}}}}^T
			\plbr{\Rtilde{2}^T\vv}\times
		\T{\Gamma}\plbr{\overline{\T{g}}}
		\T{\Gamma}^{-1}\plbr{\T{g}_{2\leftarrow{3}}}\Rtilde{2}^T
		\plbr{\T{\theta_{3\delta}}-\T{\theta_{2\delta}}}\\ 
\end{align}
The other contributions are:
\begin{align}
	\delta\T{v}_r &= \sum_{i=1}^3N_i\plbr{\delta\T{v}_i 
			- \TT{R}_i\T{b}_i\times\delta\T{\omega}_i
			- \T{\omega}_i\times\TT{R}_i\T{b}_i
			\times\T{\theta}_{\delta i}}\\
	\delta\T{\omega}_r &= \sum_{i=1}^3N_i\delta\T{\omega}_i\\ 
	\delta\T{x}_r &= \sum_{i=1}^3N_i\plbr{\delta\T{x}_i 
			- \TT{R}_i\T{b}_i\times\T{\theta}_{\delta i}}\\
	\delta\T{x}_i &= \delta\T{x}_i
\end{align}
\end{comment}

