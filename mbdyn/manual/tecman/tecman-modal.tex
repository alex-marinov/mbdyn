\subsection{Kinematics}
Position of an arbitrary point $P$
\begin{equation}
	\T{x}_P = \T{x}_0 + \T{f}_P + \T{u}_P
\end{equation}
where $\T{x}_0$ is the position of the point that describes the global
motion of the body,
$\T{f}_P$ is the relative position of the point when the body 
is undeformed,
and $\T{u}_P$ is the relative displacement of the point when the body
is deformed.

It can be rewritten as
\begin{equation}
	\T{x}_P = \T{x}_0 + \T{R}_0\plbr{\tilde{\T{f}}_P + \tilde{\T{u}}_P}
\end{equation}
where $\T{R}_0$ is the global orientation matrix of the body,
and the \emph{tilde} $\plbr{\tilde{\cdot}}$ indicates entities
expressed in the reference frame attached to the body.

The deformation of the body is expressed by a linear combination
of displacement shapes
\begin{equation}
	\tilde{\T{u}}_P = \sum_{j=1,M} \T{U}_{Pj} q_j = \T{U}_P \T{q}
\end{equation}
where $\T{U}_{Pj}$ is the vector containing the components
of the $j$-th displacement shape related to point $P$.

The orientation of the generic point $P$ is
\begin{equation}
	\T{R}_P = \T{R}_0 \tilde{\T{R}}_P
\end{equation}
and, assuming a representation of the relative orientation by a linear
combination of rotation shapes
\begin{equation}
	\tilde{\T{\phi}} = \sum_{j=1,M} \T{V}_{Pj} q_j = \T{V}_P \T{q}
\end{equation}
it results in a linearized orientation
\begin{equation}
	\T{R}_P \cong \T{R}_0 \plbr{\T{I} + \plbr{\T{V}_P \T{q}}\times{}}
\end{equation}
which is no longer orthogonal.

The first and second derivatives of position and orientation yield:
\begin{align}
	\dot{\T{x}}_P &= \dot{\T{x}}_0
		+ \T{\omega}_0 \times \T{R}_0 \plbr{\tilde{\T{f}}_P + \T{U}_P \T{q}}
		+ \T{R}_0 \T{U}_P \dot{\T{q}} \\
	\T{\omega}_P &= \T{\omega}_0
		+ \T{R}_0 \T{V}_P \dot{\T{q}} \\
	\ddot{\T{x}}_P &= \ddot{\T{x}}_0
		+ \dot{\T{\omega}}_0 \times \T{R}_0 \plbr{\tilde{\T{f}}_P + \T{U}_P \T{q}}
		+ \T{\omega}_0 \times \T{\omega}_0 \times \T{R}_0 \plbr{\tilde{\T{f}}_P + \T{U}_P \T{q}} \nonumber \\
		& \mbox{} + 2 \T{\omega}_0 \times \T{R}_0 \T{U}_P \dot{\T{q}}
		+ \T{R}_0 \T{U}_P \ddot{\T{q}} \\
	\dot{\T{\omega}}_P &= \dot{\T{\omega}}_0
		+ \T{\omega}_0 \times \T{R}_0 \T{V}_P \dot{\T{q}}
		+ \T{R}_0 \T{V}_P \ddot{\T{q}}
\end{align}

The virtual perturbation of the position and orientation
of the generic point $P$ are:
\begin{align}
	\delta{\T{x}}_P &= \delta{\T{x}}_0
		+ \delta\T{\phi}_0 \times \T{R}_0 \plbr{\tilde{\T{f}} + \T{U} \T{q}}
		+ \T{R}_0 \T{U} \delta{\T{q}} \\
	\delta\T{\phi}_P &= \delta\T{\phi}_0
		+ \T{R}_0 \T{V} \delta{\T{q}}
\end{align}

From now on, it is assumed that the structure of the problem is
given in form of lumped inertia parameters in specific points,
corresponding to FEM nodes, and that the node corresponds to the
center of mass of each lumped mass.
So the nodal mass of the $i$-th FEM node is
\begin{equation}
	\T{M}_i = \sqbr{\matr{cc}{
		m_i \T{I} & \T{0} \\
		\T{0} & \T{J}_i
	}}
\end{equation}
There is no requirement for matrix $\T{J}_i$ to be diagonal.

The inertia forces and moments acting on each FEM node are:
\begin{align}
	\T{F}_i &= - m_i \ddot{\T{x}}_i \\
	\T{C}_i &= - \T{R}_i \T{J}_i \T{R}_i^T \dot{\T{\omega}}_i
\end{align}
and the virtual work done by the inertia forces is
\begin{equation}
	\delta{L} = \sum_{i=1,N} \plbr{
		\delta{\T{x}}_i^T \T{F}_i
		+ \delta\T{\phi}_i^T \T{C}_i
	}
\end{equation}
which results in
\begin{equation}
	\delta{L} = \cubr{\cvvect{
		\delta\T{x}_0 \\
		\delta\T{\phi}_0 \\
		\delta\T{q}
	}}^T \plbr{
	\sqbr{\matr{ccc}{
		\T{M}_{xx} & \T{M}_{x\phi} & \T{M}_{xq} \\
			& \T{M}_{\phi\phi} & \T{M}_{\phi q} \\
		\llk{sym.} & & \T{M}_{qq}
	}} \cubr{\cvvect{
		\ddot{\T{x}}_0 \\
		\dot{\T{\omega}}_0 \\
		\ddot{\T{q}}
	}} + \cubr{\cvvect{
		\T{F}_x \\
		\T{F}_{\phi} \\
		\T{F}_q
	}}
	}
\end{equation}
with
\begin{align}
	\T{M}_{xx}	&= \T{I} \sum_{i=1,N} m_i \\
	\T{M}_{x\phi}	&= \T{R}_0 \sum_{i=1,N} m_i \plbr{\T{f}_i + \T{U}_i \T{q}}\times^T \T{R}_0^T \\
	\T{M}_{xq}	&= \T{R}_0 \sum_{i=1,N} m_i \T{U}_i \\
	\T{M}_{\phi\phi}&= \T{R}_0 \sum_{i=1,N} \plbr{
		m_i \plbr{\T{f}_i + \T{U}_i \T{q}}\times
		\plbr{\T{f}_i + \T{U}_i \T{q}}\times^T
		+ \tilde{\T{R}}_i \T{J}_i \tilde{\T{R}}_i^T
	} \T{R}_0^T \\
	\T{M}_{\phi q}	&= \T{R}_0 \sum_{i=1,N} \plbr{
		m_i \plbr{\T{f}_i + \T{U}_i \T{q}}\times \T{U}_i
		+ \tilde{\T{R}}_i \T{J}_i \tilde{\T{R}}_i^T \T{V}_i
	} \\
	\T{M}_{qq}	&= \sum_{i=1,N} \plbr{
		m_i \T{U}_i^T \T{U}_i
		+ \T{V}_i^T \tilde{\T{R}}_i \T{J}_i \tilde{\T{R}}_i^T \T{V}_i
	} \\
	\T{F}_x		&= \sum_{i=1,N} m_i \plbr{
		\T{\omega}_0 \times \T{\omega}_0 \times \T{R}_0 \plbr{\tilde{\T{f}_i} + \T{U}_i \T{q}}
		+ 2 \T{\omega}_0 \times \T{R}_0 \T{U}_i \dot{\T{q}} 
	} \\
	\T{F}_{\phi}	&= \sum_{i=1,N} \T{R}_0 \lplbr{
		m_i \plbr{\tilde{\T{f}_i} + \T{U}_i \T{q}}\times \plbr{
			\T{\omega}_0 \times \T{\omega}_0 \times \T{R}_0 \plbr{\tilde{\T{f}_i} + \T{U}_i \T{q}}
			+ 2 \T{\omega}_0 \times \T{R}_0 \T{U}_i \dot{\T{q}}
		}
	} \nonumber \\
			& \rplbr{
		\mbox{} + \tilde{\T{R}}_i \T{J}_i \tilde{\T{R}}_i^T \T{R}_0^T \T{\omega}_0 \times \T{R}_0 \T{V}_i \dot{\T{q}}
	} \\
	\T{F}_q		&= \sum_{i=1,N} \lplbr{
		m_i \T{U}_i^T \T{R}_0^T \lplbr{
			\T{\omega}_0 \times \T{\omega}_0 \times \T{R}_0 \plbr{\tilde{\T{f}_i} + \T{U}_i \T{q}}
			+ 2 \T{\omega}_0 \times \T{R}_0 \T{U}_i \dot{\T{q}}
		}
	} \nonumber \\
			& \rplbr{
		\mbox{} + \T{V}_i^T \tilde{\T{R}}_i \T{J}_i \tilde{\T{R}}_i^T \T{R}_0^T \T{\omega}_0 \times \T{R}_0 \T{V}_i \dot{\T{q}}
	}
\end{align}
These terms are usually simplified to neglect higher-order contributions;
all the terms that can be computed once for all, called \emph{invariants},
are collected.


\subsection{Invariants}
\begin{enumerate}
\item[1.] Total mass
\begin{equation}
	\mathrm{Inv}_1 = \sum_{i=1,N} m_i
\end{equation}
where $m_i$ is the mass of the $i$-th FEM node\footnote{Although the input
format, because of NASTRAN legacy, allows each global direction to have
a separate mass value, invariants assume that the same value is given,
and only use the one associated to component 1.}.

\item[2.] \textbf{FIXME}

\item[3.] Static coupling between rigid body and FEM node displacements
\begin{equation}
	\T{\mathrm{Inv}}_{3j} = \sum_{i=1,N} m_i \T{U}_{ij}
\end{equation}
where the portion related to the $j$-th mode is computed by summation
of the contribute of each FEM node, obtained by multiplying the FEM node
mass $m_i$ by the three components of the modal displacement $\T{U}_{ij}$
of the $j$-th mode.

\item[4.] Static coupling between rigid body rotations and FEM node 
displacements
\begin{equation}
	\T{\mathrm{Inv}}_{4j} = \sum_{i=1,N} \plbr{
		m_i \T{X}_i \times \T{U}_{ij}
		+ \T{J}_i \T{V}_{ij}
	}
\end{equation}
where the portion related to the $j$-th mode is computed by summation
of the contribute of each FEM node, obtained by multiplying the FEM node
mass $m_i$ by the cross product of the FEM node position $\T{X}_i$ 
and the three components of the modal displacement $\T{U}_{ij}$ of the
$j$-th mode.

\item[5.] Static coupling between FEM node displacements
\begin{equation}
	\T{\mathrm{Inv}}_{5jk} = \sum_{i=1,N} m_i \T{U}_{ij} \times \T{U}_{ik}
\end{equation}
where the portion related to the $j$-th mode is computed by summation
of the contribute of each FEM node, obtained by multiplying the FEM node
mass $m_i$ by the cross product of the three components of the FEM node 
$j$-th modal displacement $\T{U}_{ij}$ and the three components of the 
$k$-th modal displacement $\T{U}_{ik}$.

\item[8.]
\begin{equation}
	\T{\mathrm{Inv}}_{8j} = \sum_{i=1,N} m_i \T{X}_{i} \times \T{U}_{ij} \times{}^T
\end{equation}

\item[9.]
\begin{equation}
	\T{\mathrm{Inv}}_{9jk} = \sum_{i=1,N} m_i \T{U}_{ij} \times \T{U}_{ik} \times{}
\end{equation}

\item[10.]
\begin{equation}
	\T{\mathrm{Inv}}_{9jk} = \sum_{i=1,N} \T{V}_{ij}\times \T{J}_i
\end{equation}

\item[11.]
\begin{equation}
	\T{\mathrm{Inv}}_{4j} = \sum_{i=1,N} \T{J}_i \T{V}_{ij}
\end{equation}

\end{enumerate}
