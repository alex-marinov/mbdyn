% MBDyn (C) is a multibody analysis code.
% http://www.mbdyn.org
%
% Copyright (C) 1996-2006
%
% Pierangelo Masarati  <masarati@aero.polimi.it>
%
% Dipartimento di Ingegneria Aerospaziale - Politecnico di Milano
% via La Masa, 34 - 20156 Milano, Italy
% http://www.aero.polimi.it
%
% Changing this copyright notice is forbidden.
%
% This program is free software; you can redistribute it and/or modify
% it under the terms of the GNU General Public License as published by
% the Free Software Foundation (version 2 of the License).
% 
%
% This program is distributed in the hope that it will be useful,
% but WITHOUT ANY WARRANTY; without even the implied warranty of
% MERCHANTABILITY or FITNESS FOR A PARTICULAR PURPOSE.  See the
% GNU General Public License for more details.
%
% You should have received a copy of the GNU General Public License
% along with this program; if not, write to the Free Software
% Foundation, Inc., 59 Temple Place, Suite 330, Boston, MA  02111-1307  USA

\chapter{Results Visualization}
\label{sec:APP:OUTPUTRESULTS}
This section describes the different ways the raw output from MBDyn
can be arranged to work with third-party software for visualization.
In most of the cases, the preparation of the results can be done 
as a post-processing, starting from raw MBDyn output files.
This is true, for instance, for EasyAnim (see
\htmladdnormallink{\texttt{http://mecara.fpms.ac.be/EasyDyn/}}{http://mecara.fpms.ac.be/EasyDyn/}
for details).
In previous versions of MBDyn, the \kw{output results} statement
allowed to generate output compatible with MSC.ADAMS; provisions 
for Altair Motion View has been under preparation for long time,
but it has never been truly supported.
Anyone interested in that interface should contact MBDyn developers.
The \kw{output results} statement is now deprecated;
support for all visualization tools will be reworked
in form of postprocessing of MBDyn's raw output.

\noindent
Some post-processing preparation instructions are available
for those packages that require special handling and thus
are builtin.
\begin{itemize}
\item \textbf{MSC.ADAMS:}
must be enabled at configure time, by using the directive
\kw{--enable-adams}.
MBDyn generates a \kw{.ada} file and a \kw{.res} file.
The \kw{.ada} file must be processed by the utility \kw{ada2cmd} 
(under development), along with some user-defined data to improve
the representation of the entities (e.g.\ shapes, colors, sizes
and so on) to generate a \kw{.cmd} file.
ADAMS/View interprets the \kw{.cmd} file to build the model,
and reads the \kw{.res} file in text form (very verbose),
which contains the analysis results.
The interface is activated by the directive
\begin{verbatim}
    output results : adams
        [ , model name , " <name> " ]
        [ , velocity , { yes | no } ]
        [ , acceleration , { yes | no } ] ;
\end{verbatim}
The model name is optional, defaulting to \kw{mbdyn}.
By default, velocities and accelerations are not output; they can be
explicitly enabled by using the \kw{velocity} and \kw{acceleration}
keywords.
Note that only the \kw{dynamic} structural nodes can output 
the accelerations, and only if their output is explicitly selected,
for performance reasons.
As a consequence, only the nodes whose native output is set will add 
their acceleration to ADAMS' output.
The sequence is:
\begin{itemize}
\item compile MBDyn with ADAMS support
\item add the directive
\begin{verbatim}
    output results : adams;
\end{verbatim}
\item run the analysis
\item generate the \kw{.cmd} file from the resulting \kw{.ada} file
\item run ADAMS/View
\item import the \kw{.cmd} file
\item import the \kw{.res} file
\end{itemize}
The \kw{velocity} flag enables the output of the velocities 
of the parts; it defaults to \kw{no} because the output 
is very verbose, and they are not required to animate
the model; they can be useful to plot diagrams in the plotting
facility of ADAMS/View.

\item \textbf{EasyAnim:}
MBDyn exploits a special version of EasyAnim, based on 1.3 sources
and patched by MBDyn developers; the patch is known to work under UN*X
(Linux, essentially); it has not been tested with Windows.
It is available from the MBDyn web site, and it has submitted to EasyAnim
developers for evaluation.

the preparation of the output is done via \kw{awk}, which is invoked
by the \kw{mbdyn2easyanim.sh} script.
It uses the \kw{.log} and \kw{.mov} files, and generates a pair of
\kw{.vol} and \kw{.van} files which contain the model and the motion,
respectively.
Use
\begin{verbatim}
    mbdyn2easyanim.sh $filename
\end{verbatim}
where \kw{\$filename} is the name of the output file from MBDyn,
with no extension.

\item \textbf{Altair MotionView:}
must be enabled at configure time, by using the directive
\kw{--enable-motionview}; the specific client libraries 
are required.
It should generate a binary model and results file compatible
with MotionView.
It is activated by the directive
\begin{verbatim}
    output results : motion view ;
\end{verbatim}
No special parameters are available at the moment; 
this interface is under development.

\end{itemize}
