% $Header$
% MBDyn (C) is a multibody analysis code.
% http://www.mbdyn.org
%
% Copyright (C) 1996-2007
%
% Pierangelo Masarati  <masarati@aero.polimi.it>
%
% Dipartimento di Ingegneria Aerospaziale - Politecnico di Milano
% via La Masa, 34 - 20156 Milano, Italy
% http://www.aero.polimi.it
%
% Changing this copyright notice is forbidden.
%
% This program is free software; you can redistribute it and/or modify
% it under the terms of the GNU General Public License as published by
% the Free Software Foundation (version 2 of the License).
% 
%
% This program is distributed in the hope that it will be useful,
% but WITHOUT ANY WARRANTY; without even the implied warranty of
% MERCHANTABILITY or FITNESS FOR A PARTICULAR PURPOSE.  See the
% GNU General Public License for more details.
%
% You should have received a copy of the GNU General Public License
% along with this program; if not, write to the Free Software
% Foundation, Inc., 59 Temple Place, Suite 330, Boston, MA  02111-1307  USA

\chapter{NetCDF Output Format}
NetCDF is a format to efficiently store and retrieve data to and from
a database on file in a portable, platform-independent manner.
Further details can be found at
\begin{quote}
\htmladdnormallink{\kw{http://www.unidata.ucar.edu/software/netcdf/}}{http://www.unidata.ucar.edu/software/netcdf/}
\end{quote}
and in \cite{NETCDF-UM}.
The link reported above also describes some of the common tools
that can be used to read and manipulate the contents of the database.

The output in NetCDF format consists in a single binary file
written by the NetCDF library and intended to be read by tools
exploiting the library itself.
This document does not describe the details of NetCDF low-level format,
since this is not intended to be accessed directly by MBDyn users.
Interested readers can consult the specific documentation \cite{NETCDF-UM}.

Following the convention of NetCDF data, each datum is defined
by a variable, whose name indicates the type of datum and the entity
that generated it, organized in a tree-like fashion.

For example, the vector containing the three components 
of the position of a structural node is
\begin{verbatim}
    node.struct.<label>.X
\end{verbatim}
Each variable usually has few attributes: the units and a description.

Each level contains some datum that is intended to make the contents
of the output database as self-explanatory as possible.
For example, the level \kw{node} contains an array whose values
are the strings indicating the node types available;
each level \kw{node.<type>} contains the labels of the available nodes
for that type, and so on.
