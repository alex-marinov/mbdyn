\documentclass[12pt,dvips,fleqn,italian]{article}
% $Header$
% Copyright (C) 1996-2015 Pierangelo Masarati <masarati@aero.polimi.it>
% Dipartimento di Ingegneria Aerospaziale, Politecnico di Milano
%
% Parentesi: tonde, quadre, curly, dritte, doppie e angolari.
\newcommand{\plbr}[1]{ \left( #1 \right) }
\newcommand{\sqbr}[1]{ \left[ #1 \right] }
\newcommand{\cubr}[1]{ \left\{ #1 \right\} }
\newcommand{\shbr}[1]{ \left| #1 \right| }
\newcommand{\nrbr}[1]{ \left\| #1 \right\| }
\newcommand{\anbr}[1]{ \langle #1 \rangle }

% Parentesi solo a sinistra: tonde, quadre, curly, dritte, doppie e angolari.
\newcommand{\lplbr}[1]{ \left( #1 \right. }
\newcommand{\lsqbr}[1]{ \left[ #1 \right. }
\newcommand{\lcubr}[1]{ \left\{ #1 \right. }
\newcommand{\lshbr}[1]{ \left| #1 \right. }
\newcommand{\lnrbr}[1]{ \left\| #1 \right. }
\newcommand{\lanbr}[1]{ \langle #1 \right. }

% Parentesi solo a destra: tonde, quadre, curly, dritte,doppie e angolari.
\newcommand{\rplbr}[1]{ \left. #1 \right) }
\newcommand{\rsqbr}[1]{ \left. #1 \right] }
\newcommand{\rcubr}[1]{ \left. #1 \right\} }
\newcommand{\rshbr}[1]{ \left. #1 \right| }
\newcommand{\rnrbr}[1]{ \left. #1 \right\| }
\newcommand{\ranbr}[1]{ \left. #1 \rangle }

% Vettori verticali:
\newcommand{\vvect}[2]{ \begin{array}{ #1 } #2 \end{array} }
\newcommand{\cvvect}[1]{ \begin{array}{c} #1 \end{array} }
\newcommand{\lvvect}[1]{ \begin{array}{l} #1 \end{array} }
\newcommand{\rvvect}[1]{ \begin{array}{r} #1 \end{array} }

% Vettori orizzontali:
\newcommand{\hvect}[2]{ \begin{array}{ #1 } #2 \end{array} }

% Matrici:
\newcommand{\matr}[2]{ \begin{array}{ #1 } #2 \end{array} }

% Integrali: uso \intg{inf}{sup}{arg}{dvar}
\newcommand{\intg}[4]{ \int_{#1}^{#2} {#3} \ {#4} }

% Limite: uso \limt{var}{lim}{arg}
\newcommand{\limt}[3]{ \lim_{{#1} \rightarrow {#2}} {#3}}

% LogLike functions
\newcommand{\llk}[1]{\ensuremath{\mathrm{#1}}}

\newcommand{\diag}[0]{\llk{diag}}
\newcommand{\tr}[0]{\llk{tr}}
\newcommand{\sym}[0]{\llk{sym}}
\newcommand{\skw}[0]{\llk{skw}}

\newcommand{\step}[0]{\llk{step}}
\newcommand{\imp}[0]{\llk{imp}}

\newcommand{\grad}[0]{\llk{grad}}
\newcommand{\divr}[0]{\llk{div}}
\newcommand{\rot}[0]{\llk{rot}}

% In italiano ...
\newcommand{\sca}[0]{\llk{sca}}

% first, second, etc
\newcommand{\first}[0]{1\ensuremath{^{\mathrm{st}}}}    % 1^st
\newcommand{\second}[0]{2\ensuremath{^{\mathrm{nd}}}}   % 2^nd
\newcommand{\third}[0]{3\ensuremath{^{\mathrm{rd}}}}    % 3^rd
\newcommand{\rth}[0]{\ensuremath{^{\mathrm{th}}}}       %  ^th

\newcommand{\degr}[0]{\ensuremath{^{\mathrm{o}}}}

% esponenziale
\providecommand{\e}[1]{\llk{e}^{#1}}


\usepackage{babel}

\begin{document}



\section*{Vincolo ``rollercoaster''}
Il vincolo impone ad un corpo di spostarsi lungo una linea e di assumere
un'orientazione imposta.
I due problemi possono essere visti separatamente.



\subsection*{Vincolo in posizione}
Il luogo delle configurazioni che il corpo pu\`{o} assumere \`{e}
parametrizzato in funzione di un'ascissa curvilinea $ s $; si richiede che
sia funzione continua fino alla derivata prima.
La posizione \`{e} data dalle coordinate $ x\plbr{s}:B\in\Re\mapsto\Re^3 $,
mentre la giacitura \`{e} definita da una matrice di rotazione
$ R\plbr{s}:B\in\Re\mapsto{SO\plbr{\Re^3}} $.
Le derivate della linea e della giacitura sono definite da
$ x'\plbr{s}=\partial{x}/\partial{s} $ e da
$ \rho\plbr{s}\times{}=\partial{R}/\partial{s}R^T $.
Si consideri l'ipotesi aggiuntiva che l'asse 3 del sistema di riferimento
che definisce l'orientazione del vincolo sia diretto come la tangente alla
linea stessa.
L'equazione di vincolo si scrive come
\begin{displaymath}
    R^T \plbr{ x_b - x\plbr{s} } \ = \ 0 ,
\end{displaymath}
la quale afferma che il punto $ x_b $ appartenente al corpo $ b $ deve
giacere sulla linea $ x\plbr{s} $ ad una generica ascissa $ s $ al momento
incognita.
La reazione vincolare che il corpo ed il vincolo si scambiano, in assenza di
attrito, \`{e} diretta nelle direzioni 1 e 2 del sistema locale.
Sia quindi $ e_i $ l'$i$-esimo versore del sistema di riferimento che
definisce la giacitura del vincolo.
Le forze agenti sul corpo $ b $ sono
\begin{displaymath}
    F \ = \ e_1 v_1 + e_2 v_2 ,
\end{displaymath}
dove $ v_i $ \`{e} la $i$-esima componente della reazione vincolare nel
sistema solidale con il vincolo.
La linearizzazione delle relazioni di vincolo risulta in
\begin{displaymath}
    R^T \plbr{ \rho \times^T \plbr{ x_b - x } - x' }\Delta{s}
    + R^T \Delta{x_b} \ = \ 0
\end{displaymath}
per quanto riguarda l'equazione di vincolo, mentre per la forza si ottiene
\begin{displaymath}
    \Delta{F} \ = \ 
        \rho \times \plbr{ e_1 v_1 + e_2 v_2 } \Delta{s}
	+ e_1 \Delta{v_1} + e_2 \Delta{v_2} .
\end{displaymath}
L'equazione di vincolo va intesa nel modo seguente: le prime due equazioni
affermano che la proiezione della distanza tra il corpo ed il vincolo nelle
direzioni normali alla linea di riferimento devono essere nulle. 
Ad una violazione di queste equazioni corrisponde la nascita di una reazione
vincolare.
La terza equazione afferma che anche la proiezione lungo la linea deve
essere nulla; ad una sua violazione per\`{o} non corrisponde una reazione
vincolare, ma essa viene usata per determinare l'ascissa curvilinea a cui il
corpo si trova.
Il sistema e significativamente lo Jacobiano si semplificano se le equazioni
di vincolo vengono scritte direttamente nel sistema globale, quindi non
premoltiplicate per la trasposta della giacitura del vincolo; in tal caso si
ottiene
\begin{displaymath}
    x_b - x\plbr{s} \ = \ 0
\end{displaymath}
e 
\begin{displaymath}
    \Delta{x_b} - x' \Delta{s} \ = \ 0 .
\end{displaymath}



\subsection*{Vincolo in rotazione}
Il vincolo in rotazione si ottiene imponendo che gli assi dei sistemi di
riferimento del corpo e del vincolo siano ortogonali a due a due.
Tale vincolo \`{e} del tutto analogo a quello che si scrive di solito tra
due corpi, solo che la matrice di rotazione del vincolo ora \`{e}
parametrizzata nell'ascissa curvilinea.
Si consideri ad esempio l'ortogonalit\`{a} tra due assi non-omonimi di corpo
e vincolo, $ e_{b1} $ e $ e_2 $.
L'equazione di vincolo \`{e}
\begin{displaymath}
    e_{b1}^T e_2 \ = \ 0 .
\end{displaymath}
Questa equazione d\`{a} origine ad una coppia che agisce attorno all'asse
ortogonale ai due che definiscono il vincolo, quindi
\begin{displaymath}
    M \ = \ e_{b1} \times e_2 m_3 .
\end{displaymath}
La loro linearizzazione d\`{a}
\begin{displaymath}
    \plbr{ e_{b1} \times e_2 }^T \Delta{g_b}
    +\plbr{ e_2 \times e_{b1} }^T \rho \Delta{s} \ = \ 0
\end{displaymath}
e
\begin{displaymath}
    \Delta{M} \ = \
        m_3 e_2 \times e_{b1} \times \Delta{g_b}
	- m_3 e_{b1} \times e_2 \times \rho \Delta{s}
	+ e_{b1} \times e_2 \Delta{m_3} .
\end{displaymath}
Quindi l'ascissa curvilinea viene risolta dal vincolo in posizione, mentre
il vincolo in rotazione si limita ad imporre, nello stesso punto, che anche
i sistemi di riferimento coincidano.



\subsection*{Supporto flessibile}
Si consideri ora un supporto flessibile, ovvero una trave.
Il vincolo impone che il corpo $b$ scorra lungo la linea di riferimento
della trave, e che sia forzato a seguirne l'orientazione.
Le equazioni scritte in precedenza si modificano per quanto riguarda la
descrizione del supporto e della sua legge oraria.
La posizione del supporto \`{e} definita come
\begin{displaymath}
    x\plbr{s} \ = \ N_i\plbr{s} x_i
\end{displaymath}
in funzione delle funzioni di forma $N_i\plbr{s}$ e delle posizioni nodali
$x_i$; la sua orientazione \`{e} definita dalla matrice di rotazione
\begin{displaymath}
    R\plbr{s} \ = \ R_r R_{\Delta}\plbr{ N_i\plbr{s} g_i }
\end{displaymath}
in funzione della orientazione di riferimento $R_r$ e dei parametri di
rotazione nodali $g_i$ che concorrono a descrivere la perturbazione di
rotazione incrementale $R_{\Delta}$.

\noindent
L'equazione di vincolo in posizione rimane formalmente quella del caso
precedente:
\begin{displaymath}
    x_b - N_i\plbr{s} x_i \ = \ 0 ;
\end{displaymath}
la sua linearizzazione, tuttavia, chiama in causa anche i gradi di
libert\`{a} dell'elemento di trave.
Si ottiene:
\begin{displaymath}
    \Delta{x_b} - N_i\plbr{s} \Delta{x_i} - N'\plbr{s} x_i \Delta{s} \ = \ 0 .
\end{displaymath}

\noindent
Si consideri una formulazione del modello di trave a volumi finiti
\cite{FV-AIAA}.
Secondo tale schema, l'elemento di trave \`{e} suddiviso nettamente in
regioni a cavallo dei nodi; le forze che cadono all'interno di una regione
sono attribuite totalmente al nodo relativo.
Quindi, dette $w_i$ le funzioni di attivazione delle regioni che
costituiscono l'elemento di trave (costanti a tratti), la reazione vincolare
viene applicata come
\begin{eqnarray*}
    F_i & = & - w_i F , \\
    M_i & = & - w_i \plbr{ x\plbr{s} - x_i }\times F 
        - \plbr{ R \times R } m .
\end{eqnarray*}
La sua linearizzazione, considerando le approssimazioni della
linearizzazione delle rotazioni ammesse dall'approccio aggiornato-aggiornato
descritto in \cite{PHD-MASARATI}, comporta i termini
\begin{eqnarray*}
    \Delta{F_i} & = & - w_i \lplbr{
        \rho \times \plbr{ e_1 v_1 + e_2 v_2 } \Delta{s}
    } \\
    & & \rplbr{ \mbox{}
	- \plbr{ e_1 v_1 + e_2 v_2 } \times N_i \Delta{g}
	+ e_1 \Delta{v_1} + e_2 \Delta{v_2}
    } , \\
    \Delta{M_i} & = & - w_i \lplbr{
        - F \times \plbr{ N_j - \delta_{ij} }\Delta{ x_j }
	+ \plbr{ x\plbr{s} - x_i }\times \Delta{F}
    } \\
    & & \rplbr{ \mbox{}
        +
    } .
\end{eqnarray*}




\section*{Conclusioni}
Il vincolo \`{e} relativamente semplice da implementare, come verificato
scrivendo un modulo per MBDyn. 
Il problema fondamentale \`{e} disporre di posizione, direzione tangente,
orientazione e curvatura del vincolo in funzione continua e derivabile
dell'ascissa curvilinea.



\bibliographystyle{ieeetr}
\bibliography{mybib}

\end{document}
